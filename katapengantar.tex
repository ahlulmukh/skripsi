\chapter*{KATA PENGANTAR}
\addcontentsline{toc}{chapter}{KATA PENGANTAR}
\noindent

Segala puji serta syukur senantiasa penulis panjatkan kepada Allah SWT 
yang telah memberikan rahmat dan petunjuk-Nya kepada penulis sehingga dapat 
menyelesaikan skripsi yang berjudul “\judulId”. Shalawat serta salam kepada baginda besar Rasulullah 
Muhammad SAW, keluarga, serta kepada para sahabat yang telah berjasa dalam 
memperjuangkan islam dan ilmu pengetahuan. Skripsi ini disusun sebagai salah 
satu syarat untuk menyelesaikan pendidikan Sarjana Terapan Program Studi 
Teknologi Rekayasa Komputer Jaringan Jurusan Teknologi Informasi dan 
Komputer Politeknik Negeri Lhokseumawe.

Dalam penyusunan skripsi ini penulis banyak mendapat dukungan, perhatian, 
dorongan, bimbingan dan berbagai bantuan dari banyak pihak terutama pada
dosen pembimbing. Dengan demikian, segala hormat dan kerendahan hati penulis 
ingin menyampaikan rasa terima kasih yang tidak terhingga terutama kepada :

\begin{enumerate}
    \item Ibu Rosnilawaty, Ayah Alm.Muhammad Jamin dan saudara kandung serta seluruh keluarga besar yang paling penulis sayangi,terima kasih telah memberikan banyak 
    dukungan dan doa sehingga penulis dapat menghadapi segala rintangan 
    dalam menyelesaikan Skripsi ini.
    \item Bapak \pembimbingUtama, selaku dosen pembimbing pertama saya 
    yang telah banyak meluangkan waktu dan tenaga untuk memberikan 
    bimbingan, membantu, mengarahkan, dan memberikan masukan sehingga 
    penulis dapat menyelesaikan Skripsi.
    \item  Bapak \pembimbingPendamping, selaku dosen pembimbing kedua saya yang telah 
    banyak meluangkan waktu dan tenaga untuk memberikan bimbingan, 
    membantu, mengarahkan, dan memberikan masukan sehingga penulis dapat 
    menyelesaikan Skripsi.
    \item Bapak Muhammad Arhami, S.Si., M.Kom selaku Ketua Jurusan Teknologi 
    Informasi dan Komputer, Politeknik Negeri Lhokseumawe.
    \item Bapak Fachri Yanuar Rudi F, M.T selaku Ketua Program Studi Teknologi 
    Rekayasa Komputer Jaringan.
    \item Bapak/Ibu dosen penguji, yang telah berkenan menguji hasil penelitian dari 
    penulis, dan memberikan kritik, saran, dan masukan yang berarti bagi penulis 
    agar penulis dapat menjadi lebih baik untuk kedepannya.
    \item Seluruh dosen pengajar dan staf administrasi Program Studi Teknologi 
    Rekayasa Komputer Jaringan, Jurusan Informasi dan Komputer yang telah 
    banyak membagi ilmu dan pengalaman berharga.
    \item Teman-teman hebat di kehidupan penulis, Fauzan Zikra, Mujibullah, Fatahillah, Reza Riski, M. Rizki Afrizal Dan Yohal Fata, yang 
    selalu sabar menghadapi penulis ketika sedang sedih dan jenuh.
    \item  Teman-teman penulis TRKJ 1.B sampai TRKJ 4.B di masa perkuliahan yang 
    terus berjuang bersama untuk menyelesaikan skripsi ini.
    \item Seluruh teman-teman seperjuangan angkatan 2019 yang telah berjuang 
    bersama untuk menyelesaikan skripsi.
    \item Kepada semua pihak yang telah membantu selama penulis mengerjakan 
    skripsi.
    
\end{enumerate}
\newpage
Dalam penyusunan Skripsi ini, penulis menyadari masih banyak kekurangan 
dalam penulisan serta pembuatan program. Oleh karena itu, penulis memohon 
maaf dan mengharapkan saran dan masukkan yang membangun kepada penulis 
sehingga bisa terus berkarya lebih baik lagi. Semoga skripsi ini dapat bermanfaat 
untuk yang memerlukannya dan semoga Allah selalu terus memberikan rahmat 
dan hidayat-Nya kepada kita semua. Aamiin Ya Rabbal Alamin.

\vspace*{2cm}
\hspace*{8cm} % Ini akan mendorong minipage ke sebelah kanan
\begin{minipage}[t]{0.45\linewidth} % Sesuaikan lebar minipage sesuai kebutuhan % Mengatur penempatan teks di dalam minipage
    \noindent
    Lhokseumawe, 28 Agustus 2023\\Penulis,
    
    \vspace*{2cm}
    \textbf{\mahasiswa} \\
    NIM: \nim
\end{minipage}