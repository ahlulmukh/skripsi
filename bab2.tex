\chapter{TINJAUAN PUSTAKA}

\section{\textit{State of the Art}}
\noindent

\textit{State of the Art} Dalam penyusunan penilitian ini, peniliti mengambil beberapa referensi terdahulu sebagai panduan penulis untuk penilitian yang dilakukan, yang kemudian  akan menjadi acuan dan perbedaan dari penilitian yang akan dilakukan dengan penilitian sebelumnya. Pemaparan \textit{State of the Art} dapat dilihat pada tabel \ref{tab:stateofart} berikut.

Ada lima penelitian terdahulu yang dianggap sebagai state of the art dalam penelitian ini. Penelitian-penelitian ini dijadikan sebagai dasar acuan yang membantu dalam mengidentifikasi persamaan dan perbedaan antara penelitian ini dan penelitian sebelumnya yang terkait. Judul penelitian ini adalah "\judulId"

\begin{landscape}
	\pagestyle{empty}
    \small % Mengurangi ukuran teks menjadi kecil
    \singlespacing
    \begin{longtable}{|p{1cm}|p{3cm}|p{4cm}|p{2cm}|p{5cm}|p{2.5cm}|p{2.5cm}|}
        \caption{State of the Art}
        \label{tab:stateofart}\\
        \hline
        No & Penulis/Tahun & Judul Artikel & Metode yang digunakan & Hasil yang diperoleh & Persamaan & Perbedaan \\ \hline
        \endfirsthead
        %
        \endhead
        %
        1 & Ibnu Ramadhan, Agung Purwanto, dan Nurahman (2020) & Pengembangan Teknologi Game Indonesia Untuk Permainan FPS 3D Multiplayer “CODE TO SHOOT” Menggunakan Unity Network (UNET) Berbasis Mobile & Unity Network & Game ini dapat dimainkan secara multiplayer tanpa perlu memasukan alamat IP karena fitur uNet dapat bekerja dengan baik. Selain itu game ini juga sudah dapat dimainkan menggunakan platform mobile (Android) & Pembuatan Game Bergenre FPS & Perbedaan Jaringan yang digunakan \\ \hline
        2 & Shena Star Sarwodi, Wibisono Sukmo Wardhono, Muhammad Aminul Akbar (2020) & Penerapan Multiplayer Pada Gim Tower Defense Menggunakan Photon Unity & Photon Unity Networking & Dengan menerapkan Photon Unity Networking pada gim tower defense maka dapat diimplem entasikan sebuah fitur yang dapat meningkatk an interaktivitas dan ketertarikan pemain pada gim yaitu fitur multiplayer. & Menggunakan Photon Unity Networking & Perbedaan diterapkan pada game yang berbeda \\ \hline
        3 & Ryan Nanda Pratama, Anton Siswo Raharjo Ansori, Ashri Dinimaharawati (2021) & Pembuatan Multiplayer Game Ucing Beling Menggunakan Asset Store Mirror & Asset Mirror & Pada multiplayer game Ucing Beling dapat dimainkan secara realtime dan berjalan dengan sesuai yang diharapkan. & Dimainkan secara realtime dan multiplayer & Peneliti menggunakan asset mirror \\ \hline
		4 & I Kadek Budi Suartama, I Gede Mahendra Darmawiguna, dan I Made Putrama (2020) & Pengembangan Game Multiplayer Pengenalan Budaya Gebug Ende Seraya Karangasem Berbasis Android & Metode pengembangan dalam penelitian ini menggunakan GDLC (Game Development Life Cycle) & pengujian blackbox mendapatkan hasil bahwa semua fungsi dan fitur-fitur yang ada dapat berjalan dengan baik dan sebagaimana mestinya. & Persamaannya yaitu menggunakan game engine unity & Perbedaannya terdapat pada metode yang digunakan. \\ \hline
		5 & Muhammad Faisal Fathurrohman Dan Iskandar Ikbal (2018) & PEMBANGUNAN GAME MULTIPLAYER EDUKASI GO GREEN 3D BERBASIS ANDROID & Google Play Games Realtime Multiplayer & Dapat terhubung secara multiplayer menggunakan google play games realtime multiplayer & ersamaannya yaitu sama sama base multiplayer & Perbedaannya terdapat pada google play games realtime multiplayer. \\ \hline
	\end{longtable}
\end{landscape}

\doublespacing
\begin{sloppypar}
\subsection{Penelitian Terdahulu Pertama}
\noindent

Penelitian terdahulu pertama diambil dari jurnal dengan judul "Pengembangan Teknologi Game Indonesia Untuk Permainan FPS 3D Multiplayeer "Code To Shoot" Menggunakan Unity Network (UNET) Berbasis Mobile". Penelitian ini bertujuan untuk membuat game FPS Dengan membawakan tema lokal dari daerah Kalimantan Tengah. Pada pengujian ini peneliti melakukan bermain secara online yang dimana dengan fiture dari UNET sendiri bisa langsung dimainkan tanpa harus memasukan alamat IP.

Perbedaan penelitian ini dengan penelitian yang akan dilakukan terdapat pada penggunaaan \textit{framework}-nya, Penelitian ini menggunakan UNET (Unity Network) sedangkan penelitian yang akan dilakukan menggunakan PUN (Photon Unity Networking).

Persamaan penelitian ini dengan penelitian yang akan dilakukan adalah sama sama \textit{game} bergenre FPS dan \textit{platform} Android.

\subsection{Penelitian Terdahulu Kedua}
\noindent

Penelitian terdahulu kedua diambil dari jurnal dengan judul  "Penerapan Multiplayer Pada Gim Tower Defense Menggunakan Photon Unity". Pada penelitian ini pengujian yang dilakukan yaitu \textit{Exprience Questioner (QEQ)} untuk mengetahui perbedaan pengalaman bermain antara perempuan dan laki-laki, permupan cenderung lebih memiliki perasaan empati dan negatif terhadap lawan bermain, sedangkan pemain jenis laki laki cenderung melibatkan perilaku terhadap lawannya.

Perbedaan penelitian ini dengan penelitian yang akan dilakukan terdapat pada game yang diterapkan, pada penelitian ini game yang diterapkan yaitu game 2D dengan genre Tower Of Defense, sedangkan penelitian yang akan dilakukan yaitu game 3D bergenre FPS.

Persamaan penelitian ini dengan penelitian yang akan dilakukan adalah sama sama menggunakan PUN (Photon Unity Networking).

\subsection{Penelitian Terdahulu Ketiga}
\noindent

Penelitian terdahulu ketiga diambil dari jurnal dengan judul "Pembuatan Multiplayer Game Ucing Beling Menggunakan Asset Store Mirror". Pada penelitian ini pengujian yang dilakukan yaitu dimainkan secara bersama sama apakah permainan dimainkan secara realtime atau secara \textit{server side to side}, dari pengujian yang dilakukan permainan dapat dimainkan secara realtime dan berjalan sesuai dengan yang diharapkan.

Perbedaan penelitian ini dengan penelitian yang akan dilakukan yaitu terhadap framework yang digunakan, pada penelitian ini framework yang digunakan yaitu asset mirror.

Persamaan dari penelitian ini dengan penelitian yang akan dilakukan yaitu sama sama dapat dimainkan secara realtime.

\subsection{Penelitian Terdahulu Keempat}
\noindent

Penelitian terdahulu keempat diambil dari jurnal dengan judul "Pengembangan Game Multiplayer Pengenalan Budaya Gebug Ende Seraya Karangasem Berbasis Android". Pada penelitian ini pengujian yang dilakukan yaitu blackboxnya saja, sebagai pengujian dari penggunaan metode GDLC (\textit{Game Development Life Cycle}) untuk memastikan apakah fitur-fitur yang telah dirancang dapat berjalan dengan semestinya.

Perbedaan penelitian ini dengan penelitian yang akan dilakukan yaitu pada \textit{game} yang dirancang, pada penelitian ini hanya sebatas merancang game sesuai metode GDLC (\textit{Game Development Life Cycle}).

Persamaan dari penelitian ini dengan penelitian yang akan dilakukan yaitu sama sama menggunakan \textit{engine} yang sama yaitu Unity, dan metode yang sama yaitu GDLC(\textit{Game Development Life Cycle}).

\subsection{Penelitian Terdahulu Kelima}
\noindent

Penelitian terdahulu kelima diambil dari jurnal dengan judul "Pembangunan Game Multiplayer Edukasi Go Green 3D Berbasis Android". Pada pengujian yang dilakukan pada penelitian yaitu untuk menguji multiplayer pada game tersebut, pada pengujian yang dilakukan multiplayer berhasil dimainkan secara realtime.

Perbedaan penelitian ini dengan penelitian yang dilakukan yaitu perbeedaan \textit{framework}, pada penelitian ini menggunakan \textit{Google Play Games Realtime Multiplayer}.

Persamaan penelitian ini dengan penelitian yang akan dilakukan sama sama base game multiplayer.
\section{Tinjauan Pustaka}
\subsection{Unity}
\noindent

Unity merupakan salah satu \textit{game} engine paling populer saat ini. Penggunaan Unity dapat digunakan untuk mengembangkan konten interaktif seperti video \textit{game}, 
visualisasi arsitektur, dan real-time 3D animasi. Unity menggunakan bahasa pemograman JavaScript dan 
C\# \cite{Ansori}. Unity juga merupakan perangkat lunak yang digunakan untuk mengembangkan \textit{game} \textit{multiplatform} yang didesain secara user \textit{friendly} 
(Iman, 2017). Keunggulan Unity adalah Unity 
dapat dengan mudah mengontrol objek-objek 
dalam gim atau aplikasi. Unity terdapat 2 jenis 
lisensi yaitu \textit{personal edition} yang dapat diakses 
secara gratis dan \textit{professional edition} yang 
diharuskan untuk membayar perbulan untuk 
mengaksesnya dengan beberapa fitur tambahan 
yang tidak terdapat di \textit{personal edition} \cite{Sarwodi}. 

\subsection{\textit{Multiplayer}}
\noindent

\textit{Multiplayer} merupakan fitur pada \textit{game} dimana pemain bermain dengan lebih dari 1 orang yang bermain 
di lingkungan \textit{game} yang sama dan waktu yang bersamaan. \textit{Game} \textit{Multiplayer} biasanya memberikan pilihan pada 
pemain untuk berbagi sumber daya sistem \textit{game} atau menggunakan internet untuk bermain bersama dalam jarak 
jauh. \textit{Game} \textit{Multiplayer} yang terhubung dengan internet melibatkan pemain yang saling terhubung melalui server. 
Sedangkan \textit{Game} \textit{Multiplayer} dengan koneksi lokal yaitu, pemain saling terhubung secara langsung dengan 
pemain lainnya, pemain terkoneksi menggunakan jaringan peer to peer. Pada \textit{Game} \textit{Multiplayer} online memiliki 
beberapa jenis kategori diantaranya adalah \textit{Massively} \textit{Multiplayer} Online \textit{game} (MMO), \textit{Massively} \textit{Multiplayer} 
Online \textit{First-person Shooter} \textit{Game} (MMOFPS), \textit{Massively} \textit{Multiplayer} online \textit{Real-time Strategy} \textit{Game}
(MMORTS), \textit{Massively} \textit{Multiplayer} Online Role-playing \textit{Game}s (MMORPG), \textit{Multiplayer} Online Battle Arena
(MOBA)\cite{Ansori}. 

\subsection{Photon Unity Networking (PUN)}
\noindent

Photon adalah sebuah framework pengembangan \textit{game} \textit{multiplayer} \textit{real-time} yang cepat, ringan, dan fleksibel. Photon terdiri dari server dan beberapa SDK klien untuk platform utama.
Photon Unity Network (PUN) adalah solusi khusus Unity yang dihadirkan dengan tingkat yang lebih tinggi: matchmaking, panggilan balik yang mudah digunakan, komponen untuk sinkronisasi \textit{Game}Objects, Remote Procedure Calls (RPCs), dan fitur serupa yang memberikan awal yang baik. Di luar itu, terdapat API yang solid dan luas untuk kontrol yang lebih canggih \cite{pun}.
Berikut gambaran 
integrasi aplikasi dengan Photon Unity Networking pada Gambar \ref{fig:photonni}
\newpage
\begin{figure}[h]
	\centering
	\includegraphics[width=10cm]{arsitektur-photon.png}
	\caption{Fitur Photon Unity Networking}
	\label{fig:photonni}
\end{figure}
\subsection{\textit{First Person Shooter}(FPS)}
\noindent

\textit{First Person Shooter} (FPS) adalah salah satu jenis \textit{game} yang saat ini sangat digemari terutama kalangan \textit{game}rs muda. FPS merupakan \textit{game} yang menggunakan sudut pandang orang pertama dimana pemain akan dibuat seolah-olah menjadi karakter utama dalam \textit{game} dengan tampilan yang berpusat pada permainan disekitar senjata atau alat yang sedang digunakan \cite{fps}.

\textit{First person shooter} merupakan jenis 3D \textit{game} shooter yang menampilkan sudut pandang orang pertama dengan 
pemain yang melihat aksi melalui mata karakter permain. Tidak seperti orang ketiga yang terlihat dari bagian 
belakang atau samping, yang memungkinkan \textit{game}r untuk melihat karakter secara keseluruhan\cite{fps}.

FPS dikembangkan pada tahun 1973 melalui permainan ruang yang belum sempurna yaitu flight simulator, yang 
menampilkan sudut pandang orang pertama dengan mengarah lebih rinci ke simulator pesawat tempur, dikembangkan untuk pasukan AS pada akhir tahun 1970-an. Permainan ini tidak lagi tersedia untuk konsumen \cite{fps}.

\subsection{C\#}
\noindent

C\# (C-sharp) adalah salah satu bahasa pemograman yang menggunakan Framework .NET. Sama seperti 
bahasa lainnya, C\# memiliki aturan pada syntax dan kode-kode yang bisa digunakan dalam pembuatan aplikasi. 
C\# cocok untuk dipelajari untuk pemula karena aturan syntax-nya lebih sederhana dibandingkan bahasa 
pemograman lainnya \cite{Ansori}.

\subsection{Wireshark}
\noindent

Wireshark merupakan sebuah \textit{software} penganalisa jaringan yang paling dikenal. \textit{Software} ini 
sangat berguna dalam menyediakan jaringan dan protokol serta memberikan informasi tentang 
data yang tertangkap pada jaringan. Software wireshark dapat menganalisa transmisi paket data 
dalam jaringan, proses koneksi dan transmisi data antar komputer\cite{wireshark}.

\subsection{\textit{Quality Of Service (QoS)}}
\noindent

QoS (Quality Of Service) adalah parameter-parameter yang menjadi indicator bagus atau
tidaknya performansi dari suatu jaringan. Parameter
yang menjadi indikator dalam QoS ini meliputi 
Bandwidth, Troughput, dan \textit{Packet Loss}, delay, dan
jitter \cite{qos}. Untuk itu dilakukan analisis Quality Of
Service (QoS) pada jaringan photon cloud .Adapun 
standar pengkuran performansi dalam suatu jaringan 
yaitu TIPHON (Telecommunicationsand Internet 
Protocol Harmonization Over Networks) yang 
mengkategorikan beberapa performansi dalam 
perhintungan tertentu.

\begin{enumerate}
	\item \textit{\textit{Throughput}} \\
	\textit{Throughput} merupakan parameter QoS yang 
menunjukkan suatu kecepatan rata-rata bandwidth
yang sebenarnya, diukur dengan satuan waktu 
tertentu pada kondisi jaringan tertentu untuk 
melakukan pengiriman paket dengan ukuran tertentu 
juga. Hasil \textit{throughput} diambil dari jumlah paket 
data yang dikirim dibagi dengan jumlah waktu yang 
diperlukan saat pengiriman paket data.
	\item \textit{Packet Loss} \\
	\textit{Packet Loss} merupakan suatu parameter QoS 
yang menunjukkan suatu jumlah total keseluruhan 
paket hilang atau tidak sampai ke destinasi, 
dikarenakan adanya overload atau congestion pada 
jaringan. Dalam suatu jaringan, \textit{packet loss}
diwajibkan mempunyai persentase yang kecil sesuai 
dengan standar. 
\item \textit{Delay} \\
\textit{Delay} merupakan suatu parameter QoS yang 
menunjukkan jumlah waktu yang diperlukan paket 
untuk mencapai jarak dari source ke destination. 
Berberapa hal yang mempengaruhi delay adalah 
jarak, perangkat keras dan congestion. 
\item \textit{Jitter}\\
\textit{Jitter} merupakan suatu parameter QoS yang 
menunjukkan jumlah dari variasi-variasi delay pada 
transmisi paket pada jaringan. Hal ini disebabkan 
banyaknya variasi panjang antrian paket dalam 
waktu proses paket dan waktu penghimpunan ulang 
paket-paket.

\end{enumerate}

\subsection{\textit{Use Case}}
\noindent

Use Case atau diagram Use Case merupakan 
pemodelan untuk melakukan (Behavior) sistem 
informasi yang akan dibuat. Use Case
mendiskripsikan sebuah interaksi antara satu 
atau lebih aktor dengan sistem informasi yang 
akan dibuat. Secara kasar, Use Case digunakan 
untuk mengetahui fungsi apa saja yang ada di 
dalam sebuah sistem informasi dan siapa saja 
yang berhak menggunakan fungsi – fungsi itu. 
Syarat penamaan pada Use Case adalah nama 
didefinisikan sesimpel mungkin dan dapat 
dipahami\cite{mandiri2013pembuatan}. 

\subsection{\textit{Activity Diagram}}
\noindent

Diagram aktifitas atau activity diagram
menggambarkan workflow (aliran kerja) atau 
aktifitas dari sebuah sistem atau proses bisnis. 
Yang perlu diperhatikan disini adalah bahwa 
diagram aktivitas menggambarkan aktifitas 
sistem bukan apa yang dilakukan aktor, jadi 
aktivitas yang dapat dilakukan oleh sistem\cite{mandiri2013pembuatan}.

\end{sloppypar}





