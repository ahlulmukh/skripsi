\chapter{PENUTUP}
\section{Kesimpulan}
\noindent

Berdasarkan penelitian yang telah dilakuka oleh penulis, dapat diambil kesimpulan bahwa “Rancang Bangun Game Online FPS (Jak Meuprang) 3D Menggunakan Photon Unity Networking” berhasil dilakukan dengan kesimpulan sebagai berikut:
\begin{enumerate}
    \item Teknologi yang digunakan oleh penulis adalah Unity dengan menggunakan framework photon.
    \item Photon Unity Networking digunakan untuk memberikan permainan dapat dimainkan bersama sama dengan device yang berbeda-beda dan koneksi yang berbeda-beda.
    \item Hasil pengukuran QOS dapat disimpulkan jaringan yang disediakan oleh photon sangat bagus untuk versi gratisnya.
    \item Hasil pengujian sinkronisasi dapat disimpulkan bahwa saat player melakukan interaksi seperti bergerak,menembaki, dan waktu telah habis, maka interaksi tersebut dapat dilihat juga oleh player lainnya.
    \item Hasil pengujian dari hasil game yang telah dirancang yaitu game tersebut dapat dijalanka pada device menengah dengan fps rata rata 34FPS dan gui pada masing masing android sesuai dengan layar \textit{handphone} yang dimiliki.
\end{enumerate}

\section{Saran}
\noindent

Berdasarkan hasil dan kesimpulan yang telah disampaikan, terdapat beberapa kesimpulan inti yang dapat diambil dari penelitian ini. Selain itu, penulis juga ingin memberikan beberapa saran yang mungkin bermanfaat untuk pengembangan lebih lanjut. Adapun saran untuk pengembang game selanjutnya sebagai berikut:
\begin{enumerate}
    \item Menggunakan server Photon Unity Networking secara berbayar untuk mendapatkan pengalaman bermain secara online lebih luas dan dapat melebihi batas user yang telah ditentukan pada versi gratisnya.
    \item Menambahkan beberapa fiture senjata sebagai peningkatan. Senjata tersebut dapat ditambahkan dari berbagai senjata yang relavan dengan konteks modern.
    \item Menambahkan model karakter yang unik.
    \item Memperbaiki animationnya agar menjadi lebih baik.
\end{enumerate}