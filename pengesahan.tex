\chapter*{PENGESAHAN PEMBIMBING}
\addcontentsline{toc}{chapter}{PENGESAHAN PEMBIMBING}
\begin{sloppypar}
Skripsi yang berjudul "\judulId", disusun oleh \mahasiswa, NIM \nim, Program Studi Teknologi Rekayasa Komputer Jaringan, Jurusan Teknologi Informasi dan Komputer Politeknik Negeri Lhokseumawe telah memenuhi syarat untuk dipertanggung jawabkan didepan dewan penguji.
\end{sloppypar}

\vspace*{1cm}
\noindent 
\tabto{7.84cm}Buket Rata, 02 Agustus 2023\\
\noindent \begin{minipage}[t]{0.45\linewidth}
\noindent
Pembimbing I
  
\vspace*{2cm}
\textbf{\pembimbingUtama} \\
NIP: \nipPembimbingUtama
\end{minipage}
\hspace{0.1\linewidth}
\begin{minipage}[t]{0.45\linewidth}
  \noindent
  Pembimbing II
  
  \vspace*{2cm}
  \textbf{\pembimbingPendamping} \\
  NIP: \nipPembimbingPendamping
\end{minipage}

\vspace*{1cm}
\noindent 
\tabto{5.8cm}Mengetahui,\\
\\
\noindent \begin{minipage}[t]{0.45\linewidth}
\noindent
Ketua Jurusan\\
Teknologi Informasi dan Komputer
  
\vspace*{2cm}
\textbf{\kajur} \\
NIP: \nipKajur
\end{minipage}
\hspace{0.1\linewidth}
\begin{minipage}[t]{0.60\linewidth}
    \noindent
    Ketua Program Studi\\
    \raggedright Teknologi Rekayasa Komputer Jaringan
    
    \vspace*{2cm}
    \textbf{\kaprodi} \\
    NIP: \nipKaprodi
  \end{minipage}
  
  

