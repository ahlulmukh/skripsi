\chapter{PENDAHULUAN}
\section{Latar Belakang Masalah}
\noindent

\textit{Game} atau permainan adalah aktivitas yang dilakukan untuk tujuan hiburan atau kompetisi, dengan aturan yang telah ditentukan dan biasanya memiliki elemen interaktif yang melibatkan satu atau lebih peserta. \textit{Game} sering kali melibatkan strategi, kecepatan, keterampilan, atau ketangkasan fisik, tergantung pada jenisnya. Tujuan dari \textit{game} adalah untuk mencapai kemenangan, skor tinggi, atau hanya untuk kesenangan semata. \textit{Game} bisa dimainkan secara individu atau dalam kelompok, dan dapat berupa permainan fisik seperti sepak bola, permainan papan seperti catur atau permainan video seperti Mario Bros \parencite{fps}.

\textit{First Person Shooter} merupakan sebuah permainan peperangan menggunakan senjata api dengan sudut pandang orang pertama dan hanya menampilkan senjata yang digunakan.
Dalam permainan FPS, pemain biasanya melawan musuh secara langsung dalam pertempuran yang cepat dan intens\parencite{fps}. Senjata api menjadi alat utama pemain dalam memerangi musuh.
Agar \textit{game} \textit{First Person Shooter} (FPS) lebih menarik dimainkan, peniliti menambahkan fitur \textit{multiplayer} agar dapat dimainkan bersama sama secara online yang dapat terhubung dimana saja dengan menggunakan koneksi internet. Untuk membuat fitur \textit{multiplayer} peniliti menggunakan \textit{game} engine unity dan framework photon unity networking.

Sistem \textit{multiplayer} pada sebuah \textit{\textit{game}} membuat \textit{\textit{game}} tersebut menjadi lebih interaktif dan menarik untuk dimainkan. Dalam sebuah gim jika pemain memilih untuk single player maka pemain tersebut akan berhadapan dengan lawan NPC (Non Playable Character) sedangkan jika \textit{multiplayer} maka pemain tersebut akan berhadapan dengan pemain lain \parencite{Sarwodi}.


Penelitian ini mengusulkan sebuah \textit{\textit{game}} dengan memanfaatkan koneksi via internet yang dapat memainkan \textit{\textit{game}} bertema \textit{first person shooter}, dimana pemain bersaing secara real (nyata) dan lebih menantang di mana minimal ada 2 pemain yang akan bertemu dalam satu room.
Berdasarkan penjabaran diatas, maka diusulkan sebuah judul skripsi yang mengimplementasikan koneksi internet menggunakan photon unity asset pada \textit{\textit{game}} \textit{first person shooter} 3D yang dapat dimainkan menggunakan perangkat Android dengan judul "Rancang Bangun Game Multiplayer Online First Person Shooter(FPS) 3D Menggunakan Photon Unity Networking".
\textit{\textit{game}} ini akan dibuat \textit{multiplayer} menggunakan fitur dari unity \textit{\textit{game}} engine yaitu photon unity networking.

\section{Rumusan Masalah}
\noindent

Berdasarkan latar belakang masalah yang telah diuraikan, maka didapat perumusan masalah sebagai berikut :
\begin{enumerate}
	\item Bagaimana cara kerja fitur sinkronisasi \textit{multiplayer} online secara \textit{realtime} dengan menggunakan Photon Unity Networking dan data apa saja yang perlu disinkronisasi?
	\item Bagaimana mekanisme alur kerja \textit{first person shooter}(fps) \textit{multiplayer} dari proses persiapan bermain, mulai bermain sampai menyelesaikan permainan?
	\item Berapa persen tingkat keberhasilan pengukuran performa jaringan pada saat game dimainkan?
\end{enumerate}

\section{Batasan Masalah}
\noindent

Pada penelitian ini terdapat batasan masalah dengan maksud untuk mempermudah penulis, adapun batasan masalah pada penelitian ini sebagai berikut:
\begin{enumerate}
	\item Pembuatan \textit{\textit{game}} ini akan menggunakan IDE Unity dan bahasa pemrograman C\#.
	\item Total maksimum CCU (\textit{Concurent Users}) yang dapat terhubung ke Photon Cloud yaitu 5 CCU.
	\item Hanya dapat dimainkan diplatform Android.
	\item Hanya dapat dimainkan jika perangkat terhubung dengan koneksi internet.
	\item Tersedia sound.
	\item Tersedia senjata sebanyak 3 jenis yaitu \textit{rifle}, pistol dan pisau.
	\item Menggunakan assets open source.
	\item Tersedia Map.
	\item Terdapat dua karakter berbeda saat dimainkan.
\end{enumerate}

\section{Tujuan Penelitian}
\noindent

Adapun tujuan dari penelitian ini sebagai berikut :
\begin{enumerate}
	\item Untuk mengetahui apa saja yang disinkronisasikan pada game tersebut.
	\item Untuk mengetahui \textit{gameplay game} jak meuprang.
	\item Untuk mengetahui ke-stabilan kinerja jaringan, packet loss dan delay. 
\end{enumerate}

% \section{\textit{Rood Map}}
% \noindent

% Penelitian pertama diambil dari jurnal dengan judul "Pengembangan Game Indonesia Untuk Permainan First Person Shooter (FPS) 3D Multiplayer “CODE TO SHOOT” Menggunakan UNITY NETWORK (UNET) Berbasis Mobile".
% Penilitian ini bertujuan untuk merancang game multiplayer bergenre fps yang dapat dimainkan tanpa harus memasukan alamat ip dan game ini hanya dapat dimainkan diandroid dan tidak dapat dimainkan di laptop/pc\cite{fps}.

% Penelitian kedua diambil dari jurnal dengan judul "Penerapan Multiplayer Pada Gim Tower Defense Menggunakan Photon Unity". Penilitian ini bertujuan merancang game multiplayer bergenre strategi, yang bertujuan untuk mempertahankan wilayah atau harta benda pemain. Hasil penilitian ini berhasil menciptakan game bergenre strategi yang dapat dimainkan secara multiplayer dengan menggunakan photon unity\cite{Sarwodi}.

% Penilitian ketiga diambil dari jurnal dengan judul "Pembuatan Multiplayer Game Ucing Beling Menggunakan Asset 
% Store Mirror". Penilitian ini bertujuan merancang game multiplayer tradisional yang berasal dari jawa barat menggunakan \textit{Asset Store Mirror}. Hasil penilitan penulis menciptakan game menggunakan \textit{Asset Store Mirror}\cite{Ansori}.

% Penelitian keempat diambil dari jurnal dengan judul "Pengembangan Game Multiplayer Pengenalan 
% Budaya Gebug Ende Seraya Karangsem Berbasis 
% Android". Penilitian ini bertujuan merancang game sebagai media pengetahuan berbasis budaya yang dapat membantu masyarakat untuk lebih mengenal budaya khususnya Gebug Ende Seraya Karangsem. Hasil penilitian ini berhasil mencipatkan game yang dapat dimainkan secara multiplayer\cite{Gebug}.

% Penilitian kelima diambil dari jurnal dengan judul "Pembangunan \textit{Game} Mulitiplayer Edukasi GO GREEN 3D
% Berbasu Android". Penilitian ini bertujuan untuk merancang game dengan tema kebersihan yang dapat dimainkan secara multiplayer menggunakan Google Play Games Realtime Multiplayer. Dari jurnal ini terdapat perbedaan yaitu penulis menggunakan google play realtime multiplayer\cite{gogreen}.

\section{Manfaat Penelitian}
Manfaat dari penilitian ini antara lain adalah : 
\begin{enumerate}
	\item Memberikan hiburan dan melatih ketangkasan bermain 
	kepada pengguna.
	\item Untuk mengetahui performa jaringan photon cloud yang dimiliki photon unity networking.
	\item Sebagai bentuk implementasi konsep photon unity networking pada \textit{\textit{game}} first person shooter(fps).
\end{enumerate}

\section{Sistematika Penulisan}
\noindent

Dalam penyusunan skripsi ini, penulis memiliki sistematika penulisan agar 
penulisan skripsi ini terarah dan jelas. Adapun sistematika penulisan laporan yang 
penulis buat adalah sebagai berikut:

\vspace*{1cm}
\noindent\begin{minipage}[t]{0.2\linewidth}
	\noindent \textbf{BAB I}
\end{minipage}
\begin{minipage}[t]{0.8\linewidth}
  \noindent
  \textbf{PENDAHULUAN}\\
  Dalam bab ini menjelaskan tentang latar belakang, rumusan 
  masalah, batasan masalah, tujuan penelitian, Rood Map, manfaat 
  penelitian dan sistematika penelitian.
\end{minipage}
\\
\\
\begin{minipage}[t]{0.2\linewidth}
	\noindent \textbf{BAB II}
\end{minipage}
\begin{minipage}[t]{0.8\linewidth}
  \noindent
  \textbf{TINJAUAN PUSTAKA}\\
  Dalam bab ini menjelaskan tentang pengertian game, \textit{First Person Shooter (FPS)}, Multiplayer, Photon Unity Networking Dan \emph{Quality Of Services (QOS).}
\end{minipage}
\\
\\
\begin{minipage}[t]{0.2\linewidth}
	\noindent \textbf{BAB III}
\end{minipage}
\begin{minipage}[t]{0.8\linewidth}
  \noindent
  \textbf{METODE PENELITIAN}\\
  Dalam bab ini menjelaskan tentang rancangan dan proses yang 
  dilakukan penulis, bagian ini berisikan data dan pengumpulan data, 
  analisa, rancangan sistem (software/hardware), rancangan Use Case 
  Diagram, Story Board dan Teknik Pengujian.
\end{minipage}
\\
\\
\begin{minipage}[t]{0.2\linewidth}
	\noindent \textbf{BAB IV}
\end{minipage}
\begin{minipage}[t]{0.8\linewidth}
  \noindent
  \textbf{HASIL DAN PEMBAHASAN}\\
  Dalam bab ini menjelaskan hasil dari penelitian yang dilakukan 
  penulis, bagian ini berisikan tentang hasil, pembahasan, online dan 
  hasil pengujian.
\end{minipage}
\\
\\
\begin{minipage}[t]{0.2\linewidth}
	\noindent \textbf{BAB V}
\end{minipage}
\begin{minipage}[t]{0.8\linewidth}
  \noindent
  \textbf{PENUTUP}\\
  Bab ini akan menguraikan tentang kesimpulan dan saran dari 
penelitian ini.
\end{minipage}


