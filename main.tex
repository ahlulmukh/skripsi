\documentclass[12pt,a4paper]{report}

\usepackage{array}
\usepackage{makecell}
\usepackage{fancyhdr}
\usepackage{microtype}
\usepackage{tabto}
\newcolumntype{L}[1]{>{\raggedright\let\newline\\\arraybackslash\hspace{0pt}}m{#1}}
\newcolumntype{C}[1]{>{\centering\let\newline\\\arraybackslash\hspace{0pt}}m{#1}}
\newcolumntype{R}[1]{>{\raggedleft\let\newline\\\arraybackslash\hspace{0pt}}m{#1}}

\usepackage{mathptmx}
\usepackage{graphicx} % Required for inserting images
\graphicspath{{img/}}
\usepackage{multirow}
\usepackage{caption}
\usepackage{longtable}

\usepackage{geometry}
\geometry{
	top=4cm,
	right=3cm,
	left=4cm,
	bottom=3cm,	
}

\usepackage{pdflscape}
\usepackage{everypage}
\usepackage{tabularray}
\usepackage{listings}
\lstset{
    basicstyle=\fontsize{10}{12}\selectfont\ttfamily,
    linewidth=\dimexpr\linewidth-2cm\relax,
    breaklines=true, % Margin bingkai kanan % Atur ukuran font dan jenis huruf
}
\usepackage[table]{xcolor}
\usepackage{setspace}
\doublespacing
\usepackage{everypage}
\AddEverypageHook{\ifdim\textwidth=\linewidth\else\rotatebox{90}{\thepage}\fi}


\renewcommand{\thechapter}{\centering \Roman{chapter}}
\renewcommand{\thesection}{\arabic{chapter}.\arabic{section}}

\def\contentsname{DAFTAR ISI}
\renewcommand\bibname{DAFTAR PUSTAKA}
\def\chaptername{BAB}
\usepackage{amsmath}
\usepackage{booktabs}
\usepackage{titlesec}
\titleformat{\chapter}[display]{\normalfont\bfseries\centering}{\MakeUppercase{\chaptertitlename}~\thechapter}{0pt}{}
\titlespacing*{\chapter}{0pt}{-2pt}{16pt}

\renewcommand{\arraystretch}{1.5}

\renewcommand\thechapter{\Roman{chapter}}
\newcommand{\arabicchapter}{\arabic{chapter}}
\renewcommand\thesection{\arabic{section}}
\def\thesection{\arabic{chapter}.\arabic{section}}
\def\thetable{\arabic{table}}


\titleformat{\section}[block]{\bf\normalsize}{\thesection}{0.6em}{}
\titlespacing*{\section}{0pt}{5pt}{0pt}
\titleformat{\subsection}[block]{\bf\normalsize}{\thesubsection}{0.6em}{}
\titlespacing*{\subsection}{0pt}{5pt}{0pt}

\usepackage{setspace}
%\singlespacing
\doublespacing
%\doublespacing
%\setstretch{1.1}
\renewcommand{\tablename}{Tabel}
\renewcommand{\figurename}{Gambar}
\renewcommand{\thefigure}{\arabicchapter.\arabic{figure}}
\renewcommand{\thetable}{\arabicchapter.\arabic{table}}
\usepackage[breaklinks]{hyperref}
\renewcommand{\listfigurename}{DAFTAR GAMBAR}
\renewcommand{\listtablename}{DAFTAR TABEL}

\hyphenation{me-la-in-kan}

\newcommand{\nim}{1990343064}							% NIM Mahasiswa 
\newcommand{\mahasiswa}{Ahlul Mukhramin}	% Nama Mahasiswa
\newcommand{\judulId}{Rancang Bangun Game Online FPS (Jak Meuprang) 3D Menggunakan Photon Unity Networking }
\newcommand{\judulEn}{Thesis Title}
\newcommand{\jurusan}{Jurusan Teknologi Informasi dan Komputer}
\newcommand{\prodi}{Teknologi Rekayasa Komputer Jaringan}
\newcommand{\institusi}{Politeknik Negeri Lhokseumawe}
\newcommand{\pembimbingUtama}{Atthariq, S.ST, MT}
\newcommand{\nipPembimbingUtama}{19780724 200112 1 001}
\newcommand{\pembimbingPendamping}{Hari Toha Hidayat, S.Si., M.Cs}
\newcommand{\nipPembimbingPendamping}{19851014 201404 1 001}
\newcommand{\kajur}{Muhammad Arhami, S.Si., M,Kom}
\newcommand{\nipKajur}{19741029 200003 2 001}
\newcommand{\kaprodi}{Fachri Yanuar Rudi F, SST, MT}
\newcommand{\nipKaprodi}{19880106 201803 1 001}
\newcommand{\ketua}{Aswandi, S.Kom., M.Kom.}
\newcommand{\nipKetua}{19720924 201012 1 001}
\newcommand{\sekretaris}{Afla Nevrisa, S.Kom., M.Kom.}
\newcommand{\nipSekretaris}{19921117 202203 2 007}
\newcommand{\pembahaskedua}{Mursyidah, M.T.}
\newcommand{\nipPembahaskedua}{19730105 199903 2 003}



\begin{document}
\singlespacing
\begin{center}
    \thispagestyle{empty}
\large
\MakeUppercase{\textbf{skripsi}}

\vfill
\normalsize
Diajukan sebagai salah satu syarat untuk menyelesaikan \\
Pendidikan Jenjang Sarjana Terapan\\
Pada Politeknik Negeri Lhokseumawe.

\vfill
\begin{figure}[h]
\centering
\includegraphics[width=4cm]{logo-pnl}
\end{figure}

\vfill
\normalsize
\MakeUppercase{\textbf{\judulId}}

\vfill
Oleh: \\
\MakeUppercase{\mahasiswa} \\
\nim \\

\vfill
\MakeUppercase{
\textbf{
program studi \prodi \\
\jurusan \\
\institusi \\
\the\year{}
}}

\end{center}
\clearpage
\pagenumbering{roman}
\onehalfspacing
\chapter*{PENGESAHAN PEMBIMBING}
\addcontentsline{toc}{chapter}{PENGESAHAN PEMBIMBING}
\begin{sloppypar}
Skripsi yang berjudul "\judulId", disusun oleh \mahasiswa, NIM \nim, Program Studi Teknologi Rekayasa Komputer Jaringan, Jurusan Teknologi Informasi dan Komputer Politeknik Negeri Lhokseumawe telah memenuhi syarat untuk dipertanggung jawabkan didepan dewan penguji.
\end{sloppypar}

\vspace*{1cm}
\noindent 
\tabto{7.84cm}Buket Rata, 02 Agustus 2023\\
\noindent \begin{minipage}[t]{0.45\linewidth}
\noindent
Pembimbing I
  
\vspace*{2cm}
\textbf{\pembimbingUtama} \\
NIP: \nipPembimbingUtama
\end{minipage}
\hspace{0.1\linewidth}
\begin{minipage}[t]{0.45\linewidth}
  \noindent
  Pembimbing II
  
  \vspace*{2cm}
  \textbf{\pembimbingPendamping} \\
  NIP: \nipPembimbingPendamping
\end{minipage}

\vspace*{1cm}
\noindent 
\tabto{5.8cm}Mengetahui,\\
\\
\noindent \begin{minipage}[t]{0.45\linewidth}
\noindent
Ketua Jurusan\\
Teknologi Informasi dan Komputer
  
\vspace*{2cm}
\textbf{\kajur} \\
NIP: \nipKajur
\end{minipage}
\hspace{0.1\linewidth}
\begin{minipage}[t]{0.60\linewidth}
    \noindent
    Ketua Program Studi\\
    \raggedright Teknologi Rekayasa Komputer Jaringan
    
    \vspace*{2cm}
    \textbf{\kaprodi} \\
    NIP: \nipKaprodi
  \end{minipage}
  
  


\chapter*{PENGESAHAN PENGUJI}
\addcontentsline{toc}{chapter}{PENGESAHAN PENGUJI}
\noindent

Skripsi yang berjudul "\judulId", disusun oleh \mahasiswa, NIM \nim, Program Studi Teknologi Rekayasa Komputer Jaringan, Jurusan Teknologi Informasi dan Komputer Politeknik Negeri Lhokseumawe telah memenuhi syarat untuk dipertanggung jawabkan didepan dewan penguji dan dinyatakan lulus pada tanggal.


\vspace*{1cm}
\noindent 
\tabto{5.8cm}Dewan Penguji,\\
\noindent \begin{minipage}[t]{0.45\linewidth}
\noindent
Ketua
  
\vspace*{2cm}
\textbf{\ketua} \\
NIP: \nipKetua
\end{minipage}
\hspace{0.1\linewidth}
\begin{minipage}[t]{0.45\linewidth}
  \noindent
  Sekretaris
  
  \vspace*{2cm}
  \textbf{\sekretaris} \\
  NIP: \nipSekretaris
\end{minipage}

\vspace*{1cm}
\noindent \begin{minipage}[t]{0.43\linewidth}
\noindent
Penguji I,
  
\vspace*{2cm}
\raggedright \textbf{\kaprodi} \\
NIP: \nipKaprodi
\end{minipage}
\begin{minipage}[t]{0.345\linewidth}
  \noindent
  Penguji II,
  
  \vspace*{2cm}
  \textbf{\pembahaskedua} \\
  NIP: \nipPembahaskedua
\end{minipage}
\hspace{0.01\linewidth}
\begin{minipage}[t]{0.35\linewidth}
  \noindent
  Penguji III,
  
  \vspace*{2cm}
  \textbf{\ketua} \\
  NIP: \nipKetua
\end{minipage}


\vspace*{1cm}
\noindent 
\tabto{5.8cm}Mengetahui,\\
\\
\noindent \begin{minipage}[t]{0.45\linewidth}
\noindent
Ketua Jurusan\\
Teknologi Informasi dan Komputer
  
\vspace*{2cm}
\textbf{\kajur} \\
NIP: \nipKajur
\end{minipage}
\hspace{0.1\linewidth}
\begin{minipage}[t]{0.60\linewidth}
    \noindent
    Ketua Program Studi\\
    \raggedright Teknologi Rekayasa Komputer Jaringan
    
    \vspace*{2cm}
    \textbf{\kaprodi} \\
    NIP: \nipKaprodi
  \end{minipage}
  
  


\doublespacing
\tableofcontents
\addcontentsline{toc}{chapter}{DAFTAR ISI}

\singlespacing
\listoftables
\addcontentsline{toc}{chapter}{\listtablename}

\listoffigures
\addcontentsline{toc}{chapter}{\listfigurename}


\singlespacing
\chapter*{ABSTRAK}
\addcontentsline{toc}{chapter}{ABSTRAK}
\noindent
Perkembangan teknologi game pada software dan hardware, khususnya di android, terus
berkembang pesat setiap tahunnya. Kini, aplikasi game dalam skala 3D dapat
dijalankan dengan lancar pada android, dan integrasi antara software dan hardware semakin sem-
purna. Kinerja optimal menciptakan pengalaman bermain yang bervariasi bagi peng-
guna. Dalam peningkatan pengalaman bermain game, bermain secara bersamaan
dengan pemain lain menjadi salah satu cara untuk meningkatkan keseruan. skripsi ini merancang sebuah game bergenre FPS dengan gameplay interaktif yang memungkinkan para pemain mengendalikan karakter yang beraksi dalam dunia game. Selain itu, implementasi synchronous multiplayer memungkinkan pemain bermain secara bersamaan dalam waktu nyata, menciptakan pengalaman yang lebih mendalam dan menyenangkan.
Hasil dari pengujian rancangan game ”Jak Meuprang” menunjukkan bahwa para
pemain berhasil bermain bersama secara realtime, dan pengujian blackbox tidak
mengindikasikan masalah pada antarmuka game. Pengujian jaringan menggunakan
teknologi Photon juga menunjukkan koneksi yang stabil pada delay, memastikan pengalaman
bermain dalam mode multiplayer dapat dinikmati tanpa hambatan.
\newline
\noindent \textbf{Kata Kunci}: FPS \textit{Multiplayer}, Online \textit{Multiplayer}, \textit{Photon Unity Networking}.
\chapter*{ABSTRACT}
\addcontentsline{toc}{chapter}{ABSTRACT}
\noindent
The development of game technology in software and hardware, especially in PCs, continues
growing rapidly every year. Now, game applications in 3D scale can
run smoothly, and the integration between software and hardware is getting better
full. Optimal performance creates a varied gaming experience for users
To use. In enhancing the gaming experience, play simultaneously
with other players is one way to increase the excitement. This thesis designs an FPS genre game with interactive gameplay that allows players to control characters that act in the game world. In addition, the implementation of synchronous multiplayer allows players to play simultaneously in real time, creating a more immersive and enjoyable experience.
The test results for the game design "Jak Meuprang" show that para
players managed to play together in realtime, and the blackbox test did not
indication of a problem on the game interface. Network testing using
Photon technology also shows a stable connection on delay, ensuring the best experience
playing in multiplayer mode can be enjoyed without a hitch.
\newline
\noindent \textbf{Keywords}: FPS \textit{Multiplayer}, Online \textit{Multiplayer}, \textit{Photon Unity Networking}.
\clearpage
\doublespacing
\pagestyle{myheadings}
\makeatletter
\renewcommand{\sectionmark}[1]{\markboth{}{\bfseries}}
\pagenumbering{arabic}
% \setcounter{page}{1}
\chapter{PENDAHULUAN}
\section{Latar Belakang Masalah}
\noindent

\textit{Game} atau permainan adalah aktivitas yang dilakukan untuk tujuan hiburan atau kompetisi, dengan aturan yang telah ditentukan dan biasanya memiliki elemen interaktif yang melibatkan satu atau lebih peserta. \textit{Game} sering kali melibatkan strategi, kecepatan, keterampilan, atau ketangkasan fisik, tergantung pada jenisnya. Tujuan dari \textit{game} adalah untuk mencapai kemenangan, skor tinggi, atau hanya untuk kesenangan semata. \textit{Game} bisa dimainkan secara individu atau dalam kelompok, dan dapat berupa permainan fisik seperti sepak bola, permainan papan seperti catur atau permainan video seperti Mario Bros \parencite{fps}.

\textit{First Person Shooter} merupakan sebuah permainan peperangan menggunakan senjata api dengan sudut pandang orang pertama dan hanya menampilkan senjata yang digunakan.
Dalam permainan FPS, pemain biasanya melawan musuh secara langsung dalam pertempuran yang cepat dan intens\parencite{fps}. Senjata api menjadi alat utama pemain dalam memerangi musuh.
Agar \textit{game} \textit{First Person Shooter} (FPS) lebih menarik dimainkan, peniliti menambahkan fitur \textit{multiplayer} agar dapat dimainkan bersama sama secara online yang dapat terhubung dimana saja dengan menggunakan koneksi internet. Untuk membuat fitur \textit{multiplayer} peniliti menggunakan \textit{game} engine unity dan framework photon unity networking.

Sistem \textit{multiplayer} pada sebuah \textit{\textit{game}} membuat \textit{\textit{game}} tersebut menjadi lebih interaktif dan menarik untuk dimainkan. Dalam sebuah gim jika pemain memilih untuk single player maka pemain tersebut akan berhadapan dengan lawan NPC (Non Playable Character) sedangkan jika \textit{multiplayer} maka pemain tersebut akan berhadapan dengan pemain lain \parencite{Sarwodi}.


Penelitian ini mengusulkan sebuah \textit{\textit{game}} dengan memanfaatkan koneksi via internet yang dapat memainkan \textit{\textit{game}} bertema \textit{first person shooter}, dimana pemain bersaing secara real (nyata) dan lebih menantang di mana minimal ada 2 pemain yang akan bertemu dalam satu room.
Berdasarkan penjabaran diatas, maka diusulkan sebuah judul skripsi yang mengimplementasikan koneksi internet menggunakan photon unity asset pada \textit{\textit{game}} \textit{first person shooter} 3D yang dapat dimainkan menggunakan perangkat Android dengan judul "Rancang Bangun Game Multiplayer Online First Person Shooter(FPS) 3D Menggunakan Photon Unity Networking".
\textit{\textit{game}} ini akan dibuat \textit{multiplayer} menggunakan fitur dari unity \textit{\textit{game}} engine yaitu photon unity networking.

\section{Rumusan Masalah}
\noindent

Berdasarkan latar belakang masalah yang telah diuraikan, maka didapat perumusan masalah sebagai berikut :
\begin{enumerate}
	\item Bagaimana cara kerja fitur sinkronisasi \textit{multiplayer} online secara \textit{realtime} dengan menggunakan Photon Unity Networking dan data apa saja yang perlu disinkronisasi?
	\item Bagaimana mekanisme alur kerja \textit{first person shooter}(fps) \textit{multiplayer} dari proses persiapan bermain, mulai bermain sampai menyelesaikan permainan?
	\item Berapa persen tingkat keberhasilan pengukuran performa jaringan pada saat game dimainkan?
\end{enumerate}

\section{Batasan Masalah}
\noindent

Pada penelitian ini terdapat batasan masalah dengan maksud untuk mempermudah penulis, adapun batasan masalah pada penelitian ini sebagai berikut:
\begin{enumerate}
	\item Pembuatan \textit{\textit{game}} ini akan menggunakan IDE Unity dan bahasa pemrograman C\#.
	\item Total maksimum CCU (\textit{Concurent Users}) yang dapat terhubung ke Photon Cloud yaitu 5 CCU.
	\item Hanya dapat dimainkan diplatform Android.
	\item Hanya dapat dimainkan jika perangkat terhubung dengan koneksi internet.
	\item Tersedia sound.
	\item Tersedia senjata sebanyak 3 jenis yaitu \textit{rifle}, pistol dan pisau.
	\item Menggunakan assets open source.
	\item Tersedia Map.
	\item Terdapat dua karakter berbeda saat dimainkan.
\end{enumerate}

\section{Tujuan Penelitian}
\noindent

Adapun tujuan dari penelitian ini sebagai berikut :
\begin{enumerate}
	\item Untuk mengetahui apa saja yang disinkronisasikan pada game tersebut.
	\item Untuk mengetahui \textit{gameplay game} jak meuprang.
	\item Untuk mengetahui ke-stabilan kinerja jaringan, packet loss dan delay. 
\end{enumerate}

% \section{\textit{Rood Map}}
% \noindent

% Penelitian pertama diambil dari jurnal dengan judul "Pengembangan Game Indonesia Untuk Permainan First Person Shooter (FPS) 3D Multiplayer “CODE TO SHOOT” Menggunakan UNITY NETWORK (UNET) Berbasis Mobile".
% Penilitian ini bertujuan untuk merancang game multiplayer bergenre fps yang dapat dimainkan tanpa harus memasukan alamat ip dan game ini hanya dapat dimainkan diandroid dan tidak dapat dimainkan di laptop/pc\cite{fps}.

% Penelitian kedua diambil dari jurnal dengan judul "Penerapan Multiplayer Pada Gim Tower Defense Menggunakan Photon Unity". Penilitian ini bertujuan merancang game multiplayer bergenre strategi, yang bertujuan untuk mempertahankan wilayah atau harta benda pemain. Hasil penilitian ini berhasil menciptakan game bergenre strategi yang dapat dimainkan secara multiplayer dengan menggunakan photon unity\cite{Sarwodi}.

% Penilitian ketiga diambil dari jurnal dengan judul "Pembuatan Multiplayer Game Ucing Beling Menggunakan Asset 
% Store Mirror". Penilitian ini bertujuan merancang game multiplayer tradisional yang berasal dari jawa barat menggunakan \textit{Asset Store Mirror}. Hasil penilitan penulis menciptakan game menggunakan \textit{Asset Store Mirror}\cite{Ansori}.

% Penelitian keempat diambil dari jurnal dengan judul "Pengembangan Game Multiplayer Pengenalan 
% Budaya Gebug Ende Seraya Karangsem Berbasis 
% Android". Penilitian ini bertujuan merancang game sebagai media pengetahuan berbasis budaya yang dapat membantu masyarakat untuk lebih mengenal budaya khususnya Gebug Ende Seraya Karangsem. Hasil penilitian ini berhasil mencipatkan game yang dapat dimainkan secara multiplayer\cite{Gebug}.

% Penilitian kelima diambil dari jurnal dengan judul "Pembangunan \textit{Game} Mulitiplayer Edukasi GO GREEN 3D
% Berbasu Android". Penilitian ini bertujuan untuk merancang game dengan tema kebersihan yang dapat dimainkan secara multiplayer menggunakan Google Play Games Realtime Multiplayer. Dari jurnal ini terdapat perbedaan yaitu penulis menggunakan google play realtime multiplayer\cite{gogreen}.

\section{Manfaat Penelitian}
Manfaat dari penilitian ini antara lain adalah : 
\begin{enumerate}
	\item Memberikan hiburan dan melatih ketangkasan bermain 
	kepada pengguna.
	\item Untuk mengetahui performa jaringan photon cloud yang dimiliki photon unity networking.
	\item Sebagai bentuk implementasi konsep photon unity networking pada \textit{\textit{game}} first person shooter(fps).
\end{enumerate}

\section{Sistematika Penulisan}
\noindent

Dalam penyusunan skripsi ini, penulis memiliki sistematika penulisan agar 
penulisan skripsi ini terarah dan jelas. Adapun sistematika penulisan laporan yang 
penulis buat adalah sebagai berikut:

\vspace*{1cm}
\noindent\begin{minipage}[t]{0.2\linewidth}
	\noindent \textbf{BAB I}
\end{minipage}
\begin{minipage}[t]{0.8\linewidth}
  \noindent
  \textbf{PENDAHULUAN}\\
  Dalam bab ini menjelaskan tentang latar belakang, rumusan 
  masalah, batasan masalah, tujuan penelitian, Rood Map, manfaat 
  penelitian dan sistematika penelitian.
\end{minipage}
\\
\\
\begin{minipage}[t]{0.2\linewidth}
	\noindent \textbf{BAB II}
\end{minipage}
\begin{minipage}[t]{0.8\linewidth}
  \noindent
  \textbf{TINJAUAN PUSTAKA}\\
  Dalam bab ini menjelaskan tentang pengertian game, \textit{First Person Shooter (FPS)}, Multiplayer, Photon Unity Networking Dan \emph{Quality Of Services (QOS).}
\end{minipage}
\\
\\
\begin{minipage}[t]{0.2\linewidth}
	\noindent \textbf{BAB III}
\end{minipage}
\begin{minipage}[t]{0.8\linewidth}
  \noindent
  \textbf{METODE PENELITIAN}\\
  Dalam bab ini menjelaskan tentang rancangan dan proses yang 
  dilakukan penulis, bagian ini berisikan data dan pengumpulan data, 
  analisa, rancangan sistem (software/hardware), rancangan Use Case 
  Diagram, Story Board dan Teknik Pengujian.
\end{minipage}
\\
\\
\begin{minipage}[t]{0.2\linewidth}
	\noindent \textbf{BAB IV}
\end{minipage}
\begin{minipage}[t]{0.8\linewidth}
  \noindent
  \textbf{HASIL DAN PEMBAHASAN}\\
  Dalam bab ini menjelaskan hasil dari penelitian yang dilakukan 
  penulis, bagian ini berisikan tentang hasil, pembahasan, online dan 
  hasil pengujian.
\end{minipage}
\\
\\
\begin{minipage}[t]{0.2\linewidth}
	\noindent \textbf{BAB V}
\end{minipage}
\begin{minipage}[t]{0.8\linewidth}
  \noindent
  \textbf{PENUTUP}\\
  Bab ini akan menguraikan tentang kesimpulan dan saran dari 
penelitian ini.
\end{minipage}



\input{bab2}
\input{bab3}
\chapter{HASIL DAN PEMBAHASAN}
\section{Hasil}
\noindent

Tahapan ini merupakan tahapan yang berisi tentang penjelasan bagaimana game ini dapat bekerja sesuai dengan yang diharapkan dan dapat berjalan dengan baik. Tahapan ini meliputi perancangan perangkat lunak, bagian program yang penting dan implementasi sehingga dapat dipahami dengan baik dan mengetahui cara menggunakannya.

\subsection{Desain Perangkat Lunak}

\begin{enumerate}
    \item Tampilan \textit{Connecting Network} \\
    Tampilan \textit{connecting network} adalah tampilan awal disaat game dibuka, pada tampilan ini akan menghubungkan player ke jaringan photon .
    \begin{figure}[h]
        \centering
        \includegraphics[width=10cm]{connecting.png}
        \caption{Tampilan \textit{Connecting Network}}
        \label{fig:connecting}
    \end{figure}
    \item Tampilan Set \textit{Nickname} \\
    Set \textit{nickname menu} merupakan letak dimana player harus mengisikan nama untuk memberikan identitas nama dari player. 
    \newpage
    \begin{figure}[h]
        \centering
        \includegraphics[width=10cm]{setnama.png}
        \caption{Tampilan Mengatur Nama}
        \label{fig:setnama}
    \end{figure}
    \item Tampilan Menu Utama\\
    Tampilan menu utama adalah tampilan menu yang terdapat button "cari room" untuk mencari room yang tersedia, "buat room" untuk membuat room sebagai master room,  "thanks to" untuk melihat \textit{credit} asset yang digunakan dan "Keluar" untuk menutup game.
    \begin{figure}[h]
        \centering
        \includegraphics[width=10cm]{menuutama.png}
        \caption{Tampilan Menu Utama}
        \label{fig:menutama}
    \end{figure}
    \item Tampilan Pencarian Room\\
    Pencarian room merupakan menu untuk mencari room yang tersedia yang dibuat oleh player lain untuk memainkan game bersama.
    \newpage
    \begin{figure}[h]
        \centering
        \includegraphics[width=10cm]{pencarian.png}
        \caption{Tampilan Pencarian}
        \label{fig:pencarian}
    \end{figure}
    \item Tampilan Buat Room \\
    Buat room merupakan menu untuk membuat room bagi player ingin menjadi host pada room tersebut, pada menu buat room ini player harus mengisikan nama room terlebih dahulul seperti gambar \ref{fig:namaroom}.
    \begin{figure}[h]
        \centering
        \includegraphics[width=10cm]{namaroom.png}
        \caption{Tampilan Buat Nama Room}
        \label{fig:namaroom}
    \end{figure}
    \\ setelah player telah mengisikan nama room maka akan menampilkan \textit{instance} baru yang berisi player player pada room tersebut, dan hanya host room yang dapat memulai permainannya.
    \newpage
    \begin{figure}[h]
        \centering
        \includegraphics[width=10cm]{roomdone.png}
        \caption{Tampilan Room Telah Dibuat}
        \label{fig:roomdone}
    \end{figure}
    \item Tampilan Credit/About \\
    Tampilan ini merupakan tampilan informasi dari asset asset mana saja yang digunakan sebagai informasi.
    \begin{figure}[h]
        \centering
        \includegraphics[width=10cm]{thanksto.png}
        \caption{Tampilan About}
        \label{fig:thanksto}
    \end{figure}
    \item Tampilan Gameplay \\
    Tampilan gameplay berisi permainan yang merupakan bagian area gameplay, pada tampilan gameplay ini menampilkan \textit{head up display} (HUD) seperti tampilan health, weapon overheat, player lain dan leaderboard.
    \newpage
    \begin{figure}[h]
        \centering
        \includegraphics[width=10cm]{gameplay.png}
        \caption{Tampilan Gameplay}
        \label{fig:gameplay}
    \end{figure}
    \item Tampilan Round Over \\
    Tampilan ini ditampilkan jika waktu yang sudah ditentukan habis maka permainan telah berakhir dan player harus keluar untuk membuat room baru.
    \begin{figure}[h]
        \centering
        \includegraphics[width=10cm]{roundover.png}
        \caption{Tampilan Round Over}
        \label{fig:roundover}
    \end{figure}
\end{enumerate}

\subsection{\textit{Asset}}
\noindent

Sebuah permainan tidak luput dari berbagai asset karena asset ini adalah bahan bahan yang digunakan pada dalam pembuatan game ini untuk memperbagus tampilan dari game itu sendiri. Asset yang digunakan disini adalah asset gratis dari beberapa platform yang disediakan gratis. Berikut adalah beberapa asset utama yang digunakan dalam pembuatan game "JakMeuprang".
\newpage
\begin{enumerate}
    \item Karakter \\
    Adapun karakter yang digunakan dalam pembuatan game ini memiliki dua karakter yang memiliki animasi yang sama.
    \begin{table}[h!]
        \centering
        \caption{Karakter}\label{tbl:karakter}
        \begin{tabular}{ | c | c | m{6cm} | m{6cm} | }
            \hline
            No & Karakter & Nama \\ \hline
            1 &
            \begin{minipage}{.2\textwidth}
                \includegraphics[width=\linewidth, height=60mm, ]{karakter1.png}
            \end{minipage}
            &
            \begin{minipage}{6cm}
                Criminal Karakter
            \end{minipage}
            \\ \hline
            2 &
        \begin{minipage}{.2\textwidth}
            \includegraphics[width=\linewidth, height=60mm]{karakter2.png}
        \end{minipage}
        &
        \begin{minipage}{6cm}
            Skater Karakter
        \end{minipage}
        \\ \hline
        \end{tabular}
    \end{table}
    \newpage
    \item Weapon \\
    \begin{table}[h!]
        \centering
        \caption{Weapon}\label{tbl:weapon}
        \begin{tabular}{ | c | c | m{6cm} | m{6cm} | }
            \hline
            No & Weapon & Nama \\ \hline
            1 &
            \begin{minipage}{.2\textwidth}
                \includegraphics[width=\linewidth, height=60mm, ]{weapon1.png}
            \end{minipage}
            &
            \begin{minipage}{6cm}
                Pistol
            \end{minipage}
            \\ \hline
            2 &
        \begin{minipage}{.2\textwidth}
            \includegraphics[width=\linewidth, height=60mm]{weapon2.png}
        \end{minipage}
        &
        \begin{minipage}{6cm}
            Smg Rifle
        \end{minipage}
        \\ \hline
        3 &
        \begin{minipage}{.2\textwidth}
            \includegraphics[width=\linewidth, height=60mm]{weapon3.png}
        \end{minipage}
        &
        \begin{minipage}{6cm}
            Knife
        \end{minipage}
        \\ \hline
        \end{tabular}
    \end{table}
\end{enumerate}

\section{Pembahasan}
\noindent

Tahap ini merupakan tahap yang berisi penjelasan bagaimana proses game ini dapat bekerja sebagaimana yang diharapkan dan dapat berjalan sesuai rancangan yang telah dibuat. Pada tahap ini meliputi Proses pembuatan game, implementasi photon, gameplay, menu, karakter dan program.

\subsection{Proses Pembuatan Game}
\begin{enumerate}
    \item Pembuatan Menu\\
    Tahap pertama adalah membuat menu dimana menu ini akan berinteraksi dengan player.
    \begin{figure}[h]
        \centering
        \includegraphics[width=10cm]{pembuatanmenu.png}
        \caption{Tampilan Pembuatan Menu}
        \label{fig:pembuatanmenu}
    \end{figure}
    \item Pembuatan Player Prefabs \\
    Tahap pembuatan player prefabs ini adalah untuk memberikan player dengan karakter yang disediakan menggunakan asset gratis agar dapat digunakan saat bermain.
    \newpage
    \begin{figure}[h]
        \centering
        \includegraphics[width=10cm]{pembuatanplayer.png}
        \caption{Tampilan Pembuatan Player}
        \label{fig:pembuatanplayer}
    \end{figure}
    \item Pembuatan Animasi Karakter \\
    Pada tahap ini untuk membuat animasi dari karakter yang disediakan, dari gerakan idle dimana karakter berdiam, walking dimana saat karakter digerakan akan ada animasi berjalan, dan berlari jika inputan dari user terdeteksi berlari akan melakukan animasi berlari.
    \begin{figure}[h]
        \centering
        \includegraphics[width=10cm]{pembuatananimasi.png}
        \caption{Tampilan Pembuatan Animasi}
        \label{fig:pembuatananimasi}
    \end{figure}
    \item Objek didalam game \\ 
    Pada tahap ini menjelaskan objek objek yang terdapat saat didalam permainan
    \begin{enumerate}
        \item Weapon\\
        Player mendapatkan 3 senjata yaitu pistol, rifle, pisau.
        \item Health \\
        Player diberikan darah , jika darah berkurang makan player akan terdestroy objectnya.
        \item Weapon Overheat \\
        Player diberikan batas saat menembak, jika melebihi batas tersebut maka senjata tersebut akan overheat.
    \end{enumerate}
    \item Sound \\
    Pada tahap ini memberikan sound pada beberapa object yaitu :
    \begin{enumerate}
        \item Backsound \\
        Memberikan sound saat dalam main menu.
        \item Foot Step Low \\ 
        Memberikan sound saat player sedang berjalan maka menggunakan sound foot step low.
        \item Foot Step Fast \\
        Memberikan sound saat player berlari maka menggunakan sound foot step fast.
        \item Memberikan Sound Senjata \\ 
        Dengan memberikan sound masing masing pada senjata yang digunakan.
    \end{enumerate}
\end{enumerate}

\subsection{Map Gameplay}
Tahap ini adalah tahap pembuatan map yang digunakan saat bermain, map ini menggunakan asset low poly yang diberikan secara gratis dan dimplementasikan
\begin{figure}[h]
    \centering
    \includegraphics[width=10cm]{pembuatanmap.png}
    \caption{Tampilan Pembuatan Map}
    \label{fig:pembuatanmap}
\end{figure}

\newpage
\subsection{Program}
\noindent

Pada bagian ini akan menampilkan source code yang akan menjalankan dari interaksi interkasi object object yang telah dibuat seperti berjalan, menembak, memasuki room dan lainnya.

\begin{enumerate}
    \item Implementasi Server Photon \\
    Code ini berfungsi untuk melakukan connecting ke server photon, dengan menutup menu lainnya terlebih dahulu, jika server sudah terkoneksi maka akan masuk kedalam instance set nama dan menu.
    \begin{figure}[h]
        \centering
        \includegraphics[width=10cm]{implementasiphoton.png}
        \caption{Tampilan Code Photon}
        \label{fig:connectingp}
    \end{figure}
    \item \textit{Source Code} Create Room \\
    Pada code ini berfungsi untuk melakukan pembuatan room dengan memberikan nama room, max player, dan memanggil photonetwork untuk melakukan pembuatan room.
    \newpage
    \begin{figure}[h]
        \centering
        \includegraphics[width=10cm]{codepembuatanroom.png}
        \caption{Tampilan Code Pembuatan Room}
        \label{fig:pembuatanroom}
    \end{figure}
    \item \textit{Source Code Joinned Room} \\
    Code ini berfungsi untuk memberikan instance jika player join room akan menampikan room yang dimasuk dengan player player yang tersedia pada room tersebut.
    \begin{figure}[h]
        \centering
        \includegraphics[width=10cm]{joinnedroom.png}
        \caption{Tampilan Code Joinned Room}
        \label{fig:joinnedroom}
    \end{figure}
    \item \textit{Source Code Start Game}\\
    Code ini berfungsi untuk melakukan pemindahan instance ke scene map jika player memulai room yang telah dibuat.
    \begin{figure}[h]
        \centering
        \includegraphics[width=10cm]{startgame.png}
        \caption{Tampilan Code Start Game}
        \label{fig:startgame}
    \end{figure}
    \newpage
    \item \textit{Source Code Player Controller} \\
    Code ini berfungsi untuk memberikan instance pertama player respawn saat dimainkan
    \begin{figure}[h]
        \centering
        \includegraphics[width=10cm]{playerinstance.png}
        \caption{Tampilan Code Player Instance}
        \label{fig:playerinstance}
    \end{figure}
    \item \textit{Source Code Switch Gun} \\ 
    Code ini berfungsi untuk melakukan pergantian senjata pada player.
    \begin{figure}[h]
        \centering
        \includegraphics[width=10cm]{switchgun.png}
        \caption{Tampilan Code Switch}
        \label{fig:switchgun}
    \end{figure}
    \item \textit{Source Code Damage} \\
    Code ini berfungsi menerima damage pada player.
    \newpage
    \begin{figure}[h]
        \centering
        \includegraphics[width=10cm]{takedamage.png}
        \caption{Tampilan Code Take Damage}
        \label{fig:takedamage}
    \end{figure}
    \item \textit{Source Code Shoot} \\ 
    Code ini berfungsi untuk membuat player tembakan.
    \begin{figure}[h]
        \centering
        \includegraphics[width=10cm]{shoot.png}
        \caption{Tampilan Code Shoot}
        \label{fig:shoot}
    \end{figure}
    \item \textit{Source Code Knife} \\ 
    Code ini berfungsi untuk membuat player memberikan animasi \textit{Knife}.
    \newpage
    \begin{figure}[h]
        \centering
        \includegraphics[width=10cm]{knife.png}
        \caption{Tampilan Code Knife}
        \label{fig:knife}
    \end{figure}
    \item \textit{Source Code Movement Player} \\ 
    Code ini Berfungsi untuk membuat bergerakan player yang didapat dari inputan.
    \begin{lstlisting}
        void Update()
        {
    
            if(photonView.IsMine)
            {
    
                if (Input.GetMouseButtonDown(0))
                {
                    if (_selectedGun == 2) // Pemeriksaan jika senjata yang dipilih adalah pisau
                    {
                        KnifeAttack();
                    }
                }
                _mouseInput = new Vector2(Input.GetAxisRaw("Mouse X"), Input.GetAxisRaw("Mouse Y")) * mouseSensitivity;
    
                transform.rotation = Quaternion.Euler(transform.rotation.eulerAngles.x, 
                    transform.rotation.eulerAngles.y + _mouseInput.x ,transform.rotation.eulerAngles.z);
                _verticalRotStore += _mouseInput.y;
                _verticalRotStore = Mathf.Clamp(_verticalRotStore, -60f, 60f);
                if(invertLook){
                    viewPoint.rotation = Quaternion.Euler(_verticalRotStore, 
                        viewPoint.rotation.eulerAngles.y, viewPoint.rotation.eulerAngles.z);
                }
                else
                {
                    viewPoint.rotation = Quaternion.Euler(-_verticalRotStore, 
                        viewPoint.rotation.eulerAngles.y, viewPoint.rotation.eulerAngles.z);
                }
    
                _moveDirection = new Vector3(Input.GetAxisRaw("Horizontal"), 0f, Input.GetAxisRaw("Vertical"));
                if(Input.GetKey(KeyCode.LeftShift))
                {
                    _activeMoveSpeed = runSpeed;
    
                    if(!fast.isPlaying && _moveDirection != Vector3.zero)
                    {
                        fast.Play();
                        slow.Stop();
                    }
                }
                else
                {
                    _activeMoveSpeed = moveSpeed;
    
                    if(!slow.isPlaying && _moveDirection != Vector3.zero)
                    {
                        fast.Stop();
                        slow.Play();
                    }
                }
                if(_moveDirection == Vector3.zero || !_isGrounded)
                {
                    fast.Stop();
                    slow.Stop();
                }
                float yVal = _movement.y;
                _movement = ((transform.forward * _moveDirection.z) + (transform.right * _moveDirection.x)).normalized * _activeMoveSpeed;
                if(!charController.isGrounded)
                {
                    _movement.y = yVal;
                }
                _isGrounded = Physics.Raycast(groundCheckPoint.position, Vector3.down, .25f, groundLayers);
    
                if(Input.GetButtonDown("Jump") && _isGrounded)
                {
                    _movement.y = jumpForce;
                }
    
                _movement.y += Physics.gravity.y * Time.deltaTime * gravityMod;
    
                charController.Move(_movement * Time.deltaTime);
    
    
                if(allGuns[_selectedGun].muzzleFlash.activeInHierarchy)
                {
                    _muzzleCounter -= Time.deltaTime;
    
                    if(_muzzleCounter <= 0)
                    {
                        allGuns[_selectedGun].muzzleFlash.SetActive(false);
                    }
                }
    
                if(!_overHeated)
                {
                    if(Input.GetMouseButtonDown(0))
                    {
                        Shoot();
    
                    }
    
                    if(Input.GetMouseButton(0) && allGuns[_selectedGun].isAutomatic)
                    {
                        _shotCounter -= Time.deltaTime;
    
                        if(_shotCounter <= 0)
                        {
                            Shoot();
                        }
                    }
    
                    _heatCouner -= coolRate * Time.deltaTime;
                }
                else
                {
                    _heatCouner -= overheatCoolRate * Time.deltaTime;
    
                    if(_heatCouner <= 0)
                    {
                        _overHeated = false;
    
                        UIController.instance.overheatedMessage.gameObject.SetActive(false);
                    }
                }
    
                if(_heatCouner < 0)
                {
                    _heatCouner = 0f;
                }
    
    
                UIController.instance.weaponTempSlider.value = _heatCouner;
    
    
                if(Input.GetAxisRaw("Mouse ScrollWheel") > 0)
                {
                    _selectedGun++;
                    if(_selectedGun >= allGuns.Count)
                    {
                        _selectedGun = 0;
                    }
                    //SwitchGun();
                    photonView.RPC("SetGun", RpcTarget.All, _selectedGun);
    
                } 
                else if(Input.GetAxisRaw("Mouse ScrollWheel") < 0)
                {
                    _selectedGun--;
                    if(_selectedGun < 0)
                    {
                        _selectedGun = allGuns.Count - 1;
                    }
                    //SwitchGun();
                    photonView.RPC("SetGun", RpcTarget.All, _selectedGun);
                }
    
                for(var i = 0; i < allGuns.Count; i++)
                {
                    if(Input.GetKeyDown((i + 1).ToString()))
                    {
                        _selectedGun = i;
                        //SwitchGun();
                        photonView.RPC("SetGun", RpcTarget.All, _selectedGun);
                    }
                }
    
                
                anim.SetBool("grounded", _isGrounded);
                anim.SetFloat("speed", _moveDirection.magnitude);
    
    
                if(Input.GetMouseButton(1))
                {
                    _cam.fieldOfView =  Mathf.Lerp(_cam.fieldOfView, allGuns[_selectedGun].adsZoom, adsSpeed * Time.deltaTime);
                    gunHolder.position = Vector3.Lerp(gunHolder.position, adsInPoint.position, adsSpeed * Time.deltaTime);
                }
                else
                {
                    _cam.fieldOfView =  Mathf.Lerp(_cam.fieldOfView, 60f, adsSpeed * Time.deltaTime);
                    gunHolder.position = Vector3.Lerp(gunHolder.position, adsOutPoint.position, adsSpeed * Time.deltaTime);
                }
    
    
    
                if(Input.GetKeyDown(KeyCode.Escape))
                {
                    Cursor.lockState = CursorLockMode.None;
                    
                }
                else if(Cursor.lockState == CursorLockMode.None)
                {
                    if(Input.GetMouseButtonDown(0) && !UIController.instance.optionsScreen.activeInHierarchy)
                    {
                        Cursor.lockState = CursorLockMode.Locked;
                    }
                }
            }
        }      
    \end{lstlisting}
    \newpage
    \item \textit{Source Code Match Manager} \\
    Code ini berfungsi untuk melakukan instance match manager untuk mengatur logika selama bermain.
    \begin{figure}[h]
        \centering
        \includegraphics[width=10cm]{matchmanager.png}
        \caption{Tampilan Code Match Manger}
        \label{fig:matchmanager}
    \end{figure}
    \item \textit{Source Code Leaderboard} \\ 
    Code ini berfungsi untuk menampilkan leaderboard saat bermain.
    \begin{figure}[h]
        \centering
        \includegraphics[width=10cm]{leaderboard.png}
        \caption{Tampilan Code LeaderBoard}
        \label{fig:leaderboard}
    \end{figure}
    \item \textit{Source Code Spawn Manager}\\
    Code ini berfungsi untuk memberikan spawn player secara random dengan titik yang telah ditentukan.
    \newpage
    \begin{figure}[h]
        \centering
        \includegraphics[width=10cm]{spawnmanager.png}
        \caption{Tampilan Code Spawn Manager}
        \label{fig:spawnmanager}
    \end{figure}
    \item \textit{Source Code Player Spawner} \\ 
    Code ini berfungsi untuk melakukan instance spawn dan matinya player.
    \begin{figure}[h]
        \centering
        \includegraphics[width=10cm]{playerspawner.png}
        \caption{Tampilan Code Player Spawner}
        \label{fig:playerspawner}
    \end{figure}
    \item \textit{Source Code Gun} \\ 
    Code ini berfungsi untuk memberikan algoritma pada senajata yang digunakan.
    \newpage
    \begin{figure}[h]
        \centering
        \includegraphics[width=10cm]{gun.png}
        \caption{Tampilan Code Gun}
        \label{fig:gun}
    \end{figure}
\end{enumerate}

\section{Pengujian Multiplayer Online}

Pada pengujian multiplayer online ini terdapat dua player untuk melakukan pengujian secara online.
\begin{enumerate}
    \item Pengujian Mencari room \\
    Pada Pengujian ini player pertama mencari room yang dibuat oleh player kedua, dan pada pengujian pencarian room ini berhasil seperti gambar \ref{fig:pencarianroom}. 
    \begin{figure}[h]
        \centering
        \includegraphics[width=10cm]{pencarianroom.png}
        \caption{Tampilan Pencarian Room}
        \label{fig:pencarianroom}
    \end{figure}
    \item Pengujian Masuk Room \\
    Pada Pengujian memasuki room player pertama berhasil menjumpai player kedua didalam room tersebut seperti gambar \ref{fig:didalamroom}.
    \newpage
    \begin{figure}[h]
        \centering
        \includegraphics[width=10cm]{pengujianroom.png}
        \caption{Tampilan Didalam Room}
        \label{fig:didalamroom}
    \end{figure}
    \item Pengujian Mulai Game \\
    Pada Pengujian mulai game, player berhasil menjumpai sesama player didalam instance game secara online.
    \begin{figure}[h]
        \centering
        \includegraphics[width=10cm]{leaderboardplayer.png}
        \caption{Tampilan Dalam Game}
        \label{fig:dalamgame}
    \end{figure}
    \item Pengujian Perang \\
    Pada pengujian ini player melakukan peperangan untuk mencari point, pengujian ini berhasil dikarenakan player mati saat ditembaki oleh player lain.
    \newpage
    \begin{figure}[h]
        \centering
        \includegraphics[width=10cm]{saatdibunuh.png}
        \caption{Tampilan Dibunuh}
        \label{fig:dibunuh}
    \end{figure}
    \item Pengujian Jaringan \\
    Pada pengujian ini player dapat saling terhubung dengan perbedaan jaringan yang terkoneksi.
    \begin{figure}[h]
        \centering
        \includegraphics[width=10cm]{laptop.jpeg}
        \caption{Tampilan Jaringan Berbeda}
        \label{fig:jaringanberbeda}
    \end{figure}
\end{enumerate}



\chapter{PENUTUP}
\section{Kesimpulan}
\noindent

Berdasarkan penelitian yang telah dilakuka oleh penulis, dapat diambil kesimpulan bahwa “Rancang Bangun Game Online FPS (Jak Meuprang) 3D Menggunakan Photon Unity Networking” berhasil dilakukan dengan kesimpulan sebagai berikut:
\begin{enumerate}
    \item Teknologi yang digunakan oleh penulis adalah Unity dengan menggunakan framework photon.
    \item Photon Unity Networking digunakan untuk memberikan permainan dapat dimainkan bersama sama dengan device yang berbeda-beda dan koneksi yang berbeda-beda.
    \item Hasil pengukuran QOS dapat disimpulkan jaringan yang disediakan oleh photon sangat bagus untuk versi gratisnya.
    \item Hasil pengujian sinkronisasi dapat disimpulkan bahwa saat player melakukan interaksi seperti bergerak,menembaki, dan waktu telah habis, maka interaksi tersebut dapat dilihat juga oleh player lainnya.
    \item Hasil pengujian dari hasil game yang telah dirancang yaitu game tersebut dapat dijalanka pada device menengah dengan fps rata rata 34FPS dan gui pada masing masing android sesuai dengan layar \textit{handphone} yang dimiliki.
\end{enumerate}

\section{Saran}
\noindent

Berdasarkan hasil dan kesimpulan yang telah disampaikan, terdapat beberapa kesimpulan inti yang dapat diambil dari penelitian ini. Selain itu, penulis juga ingin memberikan beberapa saran yang mungkin bermanfaat untuk pengembangan lebih lanjut. Adapun saran untuk pengembang game selanjutnya sebagai berikut:
\begin{enumerate}
    \item Menggunakan server Photon Unity Networking secara berbayar untuk mendapatkan pengalaman bermain secara online lebih luas dan dapat melebihi batas user yang telah ditentukan pada versi gratisnya.
    \item Menambahkan beberapa fiture senjata sebagai peningkatan. Senjata tersebut dapat ditambahkan dari berbagai senjata yang relavan dengan konteks modern.
    \item Menambahkan model karakter yang unik.
    \item Memperbaiki animationnya agar menjadi lebih baik.
\end{enumerate}

\bibliographystyle{apalike} % We choose the "plain" reference style
\bibliography{reference}
\addcontentsline{toc}{chapter}{DAFTAR PUSTAKA}
\end{document}
