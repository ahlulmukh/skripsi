\chapter{HASIL DAN PEMBAHASAN}
\section{Hasil}
\noindent

Tahapan ini merupakan tahapan yang berisi tentang penjelasan bagaimana game ini dapat bekerja sesuai dengan yang diharapkan dan dapat berjalan dengan baik. Tahapan ini meliputi perancangan perangkat lunak, bagian program yang penting dan implementasi sehingga dapat dipahami dengan baik dan mengetahui cara menggunakannya.

\subsection{Desain Perangkat Lunak}

\begin{enumerate}
    \item Tampilan \textit{Connecting Network} \\
    Tampilan \textit{connecting network} adalah tampilan awal disaat game dibuka, pada tampilan ini akan menghubungkan player ke jaringan photon .
    \begin{figure}[h]
        \centering
        \includegraphics[width=10cm]{connecting.png}
        \caption{Tampilan \textit{Connecting Network}}
        \label{fig:connecting}
    \end{figure}
    \item Tampilan Set \textit{Nickname} \\
    Set \textit{nickname menu} merupakan letak dimana player harus mengisikan nama untuk memberikan identitas nama dari player. 
    \newpage
    \begin{figure}[h]
        \centering
        \includegraphics[width=10cm]{setnama.png}
        \caption{Tampilan Mengatur Nama}
        \label{fig:setnama}
    \end{figure}
    \item Tampilan Menu Utama\\
    Tampilan menu utama adalah tampilan menu yang terdapat button "cari room" untuk mencari room yang tersedia, "buat room" untuk membuat room sebagai master room,  "thanks to" untuk melihat \textit{credit} asset yang digunakan dan "Keluar" untuk menutup game.
    \begin{figure}[h]
        \centering
        \includegraphics[width=10cm]{menuutama.png}
        \caption{Tampilan Menu Utama}
        \label{fig:menutama}
    \end{figure}
    \item Tampilan Pencarian Room\\
    Pencarian room merupakan menu untuk mencari room yang tersedia yang dibuat oleh player lain untuk memainkan game bersama.
    \newpage
    \begin{figure}[h]
        \centering
        \includegraphics[width=10cm]{pencarian.png}
        \caption{Tampilan Pencarian}
        \label{fig:pencarian}
    \end{figure}
    \item Tampilan Buat Room \\
    Buat room merupakan menu untuk membuat room bagi player ingin menjadi host pada room tersebut, pada menu buat room ini player harus mengisikan nama room terlebih dahulul seperti gambar \ref{fig:namaroom}.
    \begin{figure}[h]
        \centering
        \includegraphics[width=10cm]{namaroom.png}
        \caption{Tampilan Buat Nama Room}
        \label{fig:namaroom}
    \end{figure}
    \\ setelah player telah mengisikan nama room maka akan menampilkan \textit{instance} baru yang berisi player player pada room tersebut, dan hanya host room yang dapat memulai permainannya.
    \newpage
    \begin{figure}[h]
        \centering
        \includegraphics[width=10cm]{roomdone.png}
        \caption{Tampilan Room Telah Dibuat}
        \label{fig:roomdone}
    \end{figure}
    \item Tampilan Credit/About \\
    Tampilan ini merupakan tampilan informasi dari asset asset mana saja yang digunakan sebagai informasi.
    \begin{figure}[h]
        \centering
        \includegraphics[width=10cm]{thanksto.png}
        \caption{Tampilan About}
        \label{fig:thanksto}
    \end{figure}
    \item Tampilan Gameplay \\
    Tampilan gameplay berisi permainan yang merupakan bagian area gameplay, pada tampilan gameplay ini menampilkan \textit{head up display} (HUD) seperti tampilan health, weapon overheat, player lain dan leaderboard.
    \newpage
    \begin{figure}[h]
        \centering
        \includegraphics[width=10cm]{gameplay.png}
        \caption{Tampilan Gameplay}
        \label{fig:gameplay}
    \end{figure}
    \item Tampilan Round Over \\
    Tampilan ini ditampilkan jika waktu yang sudah ditentukan habis maka permainan telah berakhir dan player harus keluar untuk membuat room baru.
    \begin{figure}[h]
        \centering
        \includegraphics[width=10cm]{roundover.png}
        \caption{Tampilan Round Over}
        \label{fig:roundover}
    \end{figure}
\end{enumerate}

\subsection{\textit{Asset}}
\noindent

Sebuah permainan tidak luput dari berbagai asset karena asset ini adalah bahan bahan yang digunakan pada dalam pembuatan game ini untuk memperbagus tampilan dari game itu sendiri. Asset yang digunakan disini adalah asset gratis dari beberapa platform yang disediakan gratis. Berikut adalah beberapa asset utama yang digunakan dalam pembuatan game "JakMeuprang".

\begin{enumerate}
    \item Karakter \\
    Adapun karakter yang digunakan dalam pembuatan game ini memiliki dua karakter yang memiliki animasi yang sama.
    
    \begin{longtblr}[caption = {\textit{Karakter}}]{
        colspec={Q[valign=b]Q[valign=b]Q[valign=h]},
        row{1}={halign=c},
        vlines,
        hlines,
      }
      No & Gambar & Nama \\
      1 & \includegraphics[scale=0.15]{karakter1} & Criminal Karakter \\
      2 & \includegraphics[scale=0.15]{karakter2} & Skater Karakter \\
    \end{longtblr}

    \newpage
    \item Weapon \\
    \begin{longtblr}[caption = {\textit{Karakter}}]{
        colspec={Q[valign=b]Q[valign=b]Q[valign=h]},
        row{1}={halign=c},
        vlines,
        hlines,
      }
      No & Gambar & Nama \\
      1 & \includegraphics[scale=0.15]{weapon1} & Pistol \\
      2 & \includegraphics[scale=0.15]{weapon2} & Rifle \\
      2 & \includegraphics[scale=0.15]{weapon3} & Knife \\
    \end{longtblr}
\end{enumerate}

\section{Pembahasan}
\noindent

Tahap ini merupakan tahap yang berisi penjelasan bagaimana proses game ini dapat bekerja sebagaimana yang diharapkan dan dapat berjalan sesuai rancangan yang telah dibuat. Pada tahap ini meliputi Proses pembuatan game, implementasi photon, gameplay, menu, karakter dan program.
\newpage
\subsection{Proses Pembuatan Game}
\begin{enumerate}
    \item Pembuatan Menu\\
    Tahap pertama adalah membuat menu dimana menu ini akan berinteraksi dengan player.
    \begin{figure}[h]
        \centering
        \includegraphics[width=10cm]{pembuatanmenu.png}
        \caption{Tampilan Pembuatan Menu}
        \label{fig:pembuatanmenu}
    \end{figure}
    \item Pembuatan Player Prefabs \\
    Tahap pembuatan player prefabs ini adalah untuk memberikan player dengan karakter yang disediakan menggunakan asset gratis agar dapat digunakan saat bermain.
    \newpage
    \begin{figure}[h]
        \centering
        \includegraphics[width=10cm]{pembuatanplayer.png}
        \caption{Tampilan Pembuatan Player}
        \label{fig:pembuatanplayer}
    \end{figure}
    \item Pembuatan Animasi Karakter \\
    Pada tahap ini untuk membuat animasi dari karakter yang disediakan, dari gerakan idle dimana karakter berdiam, walking dimana saat karakter digerakan akan ada animasi berjalan, dan berlari jika inputan dari user terdeteksi berlari akan melakukan animasi berlari.
    \begin{figure}[h]
        \centering
        \includegraphics[width=10cm]{pembuatananimasi.png}
        \caption{Tampilan Pembuatan Animasi}
        \label{fig:pembuatananimasi}
    \end{figure}
    \item Objek didalam game \\ 
    Pada tahap ini menjelaskan objek objek yang terdapat saat didalam permainan
    \begin{enumerate}
        \item Weapon\\
        Player mendapatkan 3 senjata yaitu pistol, rifle, pisau.
        \item Health \\
        Player diberikan darah , jika darah berkurang makan player akan terdestroy objectnya.
        \item Weapon Ammo \\
        Player diberikan ammo saat menembak, jika ammo telah habis maka ammo akan auto reload.
    \end{enumerate}
    \item Sound \\
    Pada tahap ini memberikan sound pada beberapa object yaitu :
    \begin{enumerate}
        \item Foot Step Low \\ 
        Memberikan sound saat player sedang berjalan maka menggunakan sound foot step low.
        \item Foot Step Fast \\
        Memberikan sound saat player berlari maka menggunakan sound foot step fast.
        \item Memberikan Sound Senjata \\ 
        Dengan memberikan sound masing masing pada senjata yang digunakan.
    \end{enumerate}
    \item Pembuatan \textit{Mobile Input} \\
    Tahap ini adalah tahap pembuatan \textit{Mobile input} yang berfungsi untuk menerima inputan \textit{button} dari user yang terdapat pada layar user. Inputannya berupa \textit{Shoot}, \textit{Aiming}, \textit{change weapon}, \textit{pause}, \textit{Move player} dan \textit{LeaderBoard input}.
    \newpage
    \begin{figure}[h]
        \centering
        \includegraphics[width=10cm]{mobile-input.png}
        \caption{Tampilan Pembuatan \textit{Mobile Input}}
        \label{fig:mobileinput}
    \end{figure}
\item Pembuatan Map \\
Tahap ini adalah tahap pembuatan map yang digunakan saat bermain, map ini menggunakan asset low poly yang diberikan secara gratis dan dimplementasikan
\begin{figure}[h]
    \centering
    \includegraphics[width=10cm]{pembuatanmap.png}
    \caption{Tampilan Pembuatan Map}
    \label{fig:pembuatanmap}
\end{figure}
\newpage
\item Pembuatan Spawn Point \\ 
Pada tahap ini, memberikan beberapa titik spawn player secara random, jika player di\textit{kill} oleh player lain, maka sistem akan memberikan titik spawn yang sudah ditentukan.
\begin{figure}[h]
    \centering
    \includegraphics[width=10cm]{pembuatan-spawn.png}
    \caption{Tampilan Pembuatan Map}
    \label{fig:pembuatanspawn}
\end{figure}
\item Pembuatan Animasi Senjata \\
Tahap ini merupakan pembuatan animasi controller pada setiap setiap senjata yang ada yaitu pisau, m14, dan pistol.
\begin{figure}[h]
    \centering
    \includegraphics[width=10cm]{animasi-knife.png}
    \caption{Tampilan Pembuatan Animasi Knife}
    \label{fig:animasiknife}
\end{figure}
\newpage
\begin{figure}[h]
    \centering
    \includegraphics[width=10cm]{animasi-pistol.png}
    \caption{Tampilan Pembuatan Animasi Pistol}
    \label{fig:animasipistol}
\end{figure}
\begin{figure}[h]
    \centering
    \includegraphics[width=10cm]{animasi-rifle.png}
    \caption{Tampilan Pembuatan Animasi Rifle}
    \label{fig:animasirifle}
\end{figure}
\end{enumerate}

% \newpage
% \subsection{Program}
% \noindent

% Pada bagian ini akan menampilkan source code yang akan menjalankan dari interaksi interkasi object object yang telah dibuat seperti berjalan, menembak, memasuki room dan lainnya.

% \begin{enumerate}
%     \item Implementasi Server Photon \\
%     Code ini berfungsi untuk melakukan connecting ke server photon, dengan menutup menu lainnya terlebih dahulu, jika server sudah terkoneksi maka akan masuk kedalam instance set nama dan menu.
%     \begin{figure}[h]
%         \centering
%         \includegraphics[width=10cm]{implementasiphoton.png}
%         \caption{Tampilan Code Photon}
%         \label{fig:connectingp}
%     \end{figure}
%     \item \textit{Source Code} Create Room \\
%     Pada code ini berfungsi untuk melakukan pembuatan room dengan memberikan nama room, max player, dan memanggil photonetwork untuk melakukan pembuatan room.
%     \newpage
%     \begin{figure}[h]
%         \centering
%         \includegraphics[width=10cm]{codepembuatanroom.png}
%         \caption{Tampilan Code Pembuatan Room}
%         \label{fig:pembuatanroom}
%     \end{figure}
%     \item \textit{Source Code Joinned Room} \\
%     Code ini berfungsi untuk memberikan instance jika player join room akan menampikan room yang dimasuk dengan player player yang tersedia pada room tersebut.
%     \begin{figure}[h]
%         \centering
%         \includegraphics[width=10cm]{joinnedroom.png}
%         \caption{Tampilan Code Joinned Room}
%         \label{fig:joinnedroom}
%     \end{figure}
%     \item \textit{Source Code Start Game}\\
%     Code ini berfungsi untuk melakukan pemindahan instance ke scene map jika player memulai room yang telah dibuat.
%     \begin{figure}[h]
%         \centering
%         \includegraphics[width=10cm]{startgame.png}
%         \caption{Tampilan Code Start Game}
%         \label{fig:startgame}
%     \end{figure}
%     \newpage
%     \item \textit{Source Code Player Controller} \\
%     Code ini berfungsi untuk memberikan instance pertama player respawn saat dimainkan
%     \begin{figure}[h]
%         \centering
%         \includegraphics[width=10cm]{playerinstance.png}
%         \caption{Tampilan Code Player Instance}
%         \label{fig:playerinstance}
%     \end{figure}
%     \item \textit{Source Code Switch Gun} \\ 
%     Code ini berfungsi untuk melakukan pergantian senjata pada player.
%     \begin{figure}[h]
%         \centering
%         \includegraphics[width=10cm]{switchgun.png}
%         \caption{Tampilan Code Switch}
%         \label{fig:switchgun}
%     \end{figure}
%     \item \textit{Source Code Damage} \\
%     Code ini berfungsi menerima damage pada player.
%     \newpage
%     \begin{figure}[h]
%         \centering
%         \includegraphics[width=10cm]{takedamage.png}
%         \caption{Tampilan Code Take Damage}
%         \label{fig:takedamage}
%     \end{figure}
%     \item \textit{Source Code Shoot} \\ 
%     Code ini berfungsi untuk membuat player tembakan.
%     \begin{figure}[h]
%         \centering
%         \includegraphics[width=10cm]{shoot.png}
%         \caption{Tampilan Code Shoot}
%         \label{fig:shoot}
%     \end{figure}
%     \newpage
%     \item \textit{Source Code Knife} \\ 
%     Code ini berfungsi untuk membuat player memberikan animasi \textit{Knife}.
%     \begin{figure}[h]
%         \centering
%         \includegraphics[width=10cm]{knife.png}
%         \caption{Tampilan Code Knife}
%         \label{fig:knife}
%     \end{figure}
%     \item \textit{Source Code Movement Player} \\ 
%     Code ini Berfungsi untuk membuat bergerakan player yang didapat dari inputan.
%     \begin{figure}[h]
%         \centering
%         \includegraphics[width=10cm]{player-movement.png}
%         \caption{Tampilan Code Player Movement}
%         \label{fig:movementp}
%     \end{figure}
%     \item \textit{Source Code Match Manager} \\
%     Code ini berfungsi untuk melakukan instance match manager untuk mengatur logika selama bermain.
%     \newpage
%     \begin{figure}[h]
%         \centering
%         \includegraphics[width=10cm]{matchmanager.png}
%         \caption{Tampilan Code Match Manger}
%         \label{fig:matchmanager}
%     \end{figure}
%     \item \textit{Source Code Leaderboard} \\ 
%     Code ini berfungsi untuk menampilkan leaderboard saat bermain.
%     \begin{figure}[h]
%         \centering
%         \includegraphics[width=10cm]{leaderboard.png}
%         \caption{Tampilan Code LeaderBoard}
%         \label{fig:leaderboard}
%     \end{figure}
%     \item \textit{Source Code Spawn Manager}\\
%     Code ini berfungsi untuk memberikan spawn player secara random dengan titik yang telah ditentukan.
%    \newpage
%     \begin{figure}[h]
%         \centering
%         \includegraphics[width=10cm]{spawnmanager.png}
%         \caption{Tampilan Code Spawn Manager}
%         \label{fig:spawnmanager}
%     \end{figure}
%     \item \textit{Source Code Player Spawner} \\ 
%     Code ini berfungsi untuk melakukan instance spawn dan matinya player.
%     \begin{figure}[h]
%         \centering
%         \includegraphics[width=10cm]{playerspawner.png}
%         \caption{Tampilan Code Player Spawner}
%         \label{fig:playerspawner}
%     \end{figure}
%     \item \textit{Source Code Gun} \\ 
%     Code ini berfungsi untuk memberikan algoritma pada senajata yang digunakan.
%     \newpage
%     \begin{figure}[h]
%         \centering
%         \includegraphics[width=10cm]{gun.png}
%         \caption{Tampilan Code Gun}
%         \label{fig:gun}
%     \end{figure}
% \end{enumerate}

\section{Pengujian Game}
\noindent

Pada tahap ini, peneliti melakukan pengujian yaitu berupa pengujain sistem, grafis, interface, sinkronisasi dan pengujian jaringan.

\subsection{Uji Coba Pengguna}
\noindent

Dilakukan pengujian terhadap pengguna untuk 
memberikan timbal balik akurasi penilaian untuk pengujian 
performa. Berikut Daftar nama penguji pada uji coba pengguna 
yang dapat dilihat pada tabel \ref{tab:pengguna-uji}.
\newpage

\begin{table}[h]
    \centering
    \caption{Daftar Pengguna Uji Coba}
    \label{tab:pengguna-uji}
    \begin{tabular}{|l|l|l|}
    \hline
    \textbf{No} & \textbf{Nama}          & \textbf{Keterangan Device} \\ \hline
    1           & Ahlul Mukhramin        & Redmi Note 8               \\ \hline
    2           & Aldi Ferdian           & Realme 7                   \\ \hline
    3           & Fauzan Zikra           & Poco x3 Pro                    \\ \hline
    4           & Muhammad Rizki Afrizal & Redmi Note 7               \\ \hline
    5           & Mujibullah             & Vivo y15                   \\ \hline
    \end{tabular}
    \end{table}

Berikut hasil penilaian pengguna terhadap Kelancaran permainan, dan Kesesuaian GUI Pada tabel \ref{tab:pengujian-hasil}.

\begin{table}[h]
    \centering
    \caption{Hasil Pengujian Pengguna}
    \label{tab:pengujian-hasil}
    \begin{tabular}{|lllllll|l|}
    \hline
    \multicolumn{1}{|l|}{}                              & \multicolumn{1}{c|}{}                                                & \multicolumn{5}{c|}{\textbf{Penilaian}}                                                                                                                                 &                                      \\ \cline{3-7}
    \multicolumn{1}{|l|}{\multirow{-2}{*}{\textbf{No}}} & \multicolumn{1}{c|}{\multirow{-2}{*}{\textbf{Keterangan Pengujian}}} & \multicolumn{1}{c|}{\textbf{1}} & \multicolumn{1}{c|}{\textbf{2}} & \multicolumn{1}{c|}{\textbf{3}} & \multicolumn{1}{c|}{\textbf{4}} & \multicolumn{1}{c|}{\textbf{5}} & \multirow{-2}{*}{\textbf{Rata Rata}} \\ \hline
    \multicolumn{1}{|l|}{1}                             & \multicolumn{1}{l|}{Kelancaran FPS Permainan}                        & \multicolumn{1}{l|}{0}          & \multicolumn{1}{l|}{0}          & \multicolumn{1}{r|}{0}          & \multicolumn{1}{l|}{0}          & 5                               & 100\%                                \\ \hline
    \rowcolor[HTML]{9B9B9B} 
    \multicolumn{2}{|l|}{\cellcolor[HTML]{9B9B9B}}                                                                             & \multicolumn{2}{l|}{\cellcolor[HTML]{9B9B9B}Sesuai}               & \multicolumn{3}{l|}{\cellcolor[HTML]{9B9B9B}Tidak}                                                  &                                      \\ \hline
    \multicolumn{1}{|l|}{2}                             & \multicolumn{1}{l|}{Kesesuaian GUI Dengan Device}                    & \multicolumn{2}{l|}{5}                                            & \multicolumn{3}{l|}{0}                                                                              & 100\%                                \\ \hline
    \multicolumn{7}{|c|}{Total Rata Rata}                                                                                                                                                                                                                                                                & 100\%                                \\ \hline
    \end{tabular}
    \end{table}

\subsection{Pengujian Grafis 3D}
\noindent

Uji ini berupa perhitungan kecepatan \textit{frame} per detik pada perangkat Android dalam me-render objeck 3D. Sebuah permainan video akan terasa lancar pada standar \textit{framerate} 24. Pengujian ini berlangsung menggunnakan permainan itu sendiri. Hasilnya dapat dilihat pada tabel \ref{tb:tabel-grafis}.
\newpage
\begin{table}[h]
    \centering
    \caption{Hasil Pengujian Grafis 3D}
    \label{tb:tabel-grafis}
    \begin{tabular}{|l|l|l|l|}
    \hline
    \multicolumn{1}{|c|}{Nama Perangkat} & \multicolumn{1}{c|}{\begin{tabular}[c]{@{}c@{}}Fps\\ Minimum\end{tabular}} & \multicolumn{1}{c|}{\begin{tabular}[c]{@{}c@{}}Fps\\ Maksimum\end{tabular}} & \multicolumn{1}{c|}{\begin{tabular}[c]{@{}c@{}}Fps\\ Rata-Rata\end{tabular}} \\ \hline
    Redmi Note 8   & 27  & 35 & 30 \\ \hline
    Vivo y15       & 29  & 30 & 29 \\ \hline
    Redmi Note 7   & 29  & 37 & 35 \\ \hline
    Realme 7    & 29  & 60 & 54 \\ \hline
    Poco x3 Pro    & 30  & 60 & 60 \\ \hline
    \multicolumn{3}{|c|}{Total rata-rata} & 34.6 \\ \hline
    \multicolumn{3}{|l|}{Presentase Terhadap Standar Kelancaran (24FPS)} & 100\%  \\ \hline
    \end{tabular}
\end{table}
\noindent


Dapat dilihat pada tabel \ref{tb:tabel-grafis} masing-masing device dengan spek yang berbeda dapat berjalan dengan lancar sesuai standar kelancaran fps yaitu 24fps. Pada hasil yang didapat fps rata rata diatas 30fps, yang dimana sudah melebihi standar fps yaitu 24, pada device yang telah diuji coba tersebut dapat memainkan permainan ini dengan lancar. Dapat dilihat pada gambar \ref{fig:pocox3pro} , pengujian dari salah satu device yaitu poco x3 pro


\begin{figure}[h]
    \centering
    \includegraphics[width=10cm]{pocox3pro.jpg}
    \caption{Tampilan Pada Device Poco X3 Pro}
    \label{fig:pocox3pro}
\end{figure}
\newpage

\newpage
\subsection{Uji Interface GUI}
\noindent

Uji coba ini berupa pengecekan kompabilty GUI di layar \textit{landscape orientatio} pada perangkat yang berbeda-beda dan 
mengacu terhadap pengujian pengguna dengan device yang 
dimiliki. Uji coba ini dilakukan dengan mengamati langsung dengan mengirimkan \textit{file} apk kepada teman/kerabat terdekat. Hasilnya dapat dilihat pada Tabel \ref{tb:tabel-interface}

\begin{table}[h]
    \centering
    \caption{Hasil Pengujian Interface}
    \label{tb:tabel-interface}
    \begin{tabular}{|c|c|}
    \hline
    Nama Perangkat & \begin{tabular}[c]{@{}c@{}}Kesesuaian\\ GUI Pada Layar\end{tabular} \\ \hline
    Redmi Note 8   & Sesuai                                                              \\ \hline
    Vivo y15       & Sesuai                                                              \\ \hline
    Redmi Note 7   & Sesuai                                                              \\ \hline
    Poco x3 Pro    & Sesuai                                                              \\ \hline
    Rog Phone 2    & Sesuai                                                              \\ \hline
    \end{tabular}
    \end{table}
\newpage
% \subsection{Pengujian Blackbox Testing}
% \noindent

% Pengujian blackbox untuk menguji fungsi sistem atau kekurangan pada 
% perangkat lunak yang diuji agar menjadi lebih baik dan dapat diminimalisir 
% terjadinya kekurangan pada sistem
% \begin{enumerate}
%     \item Tombol Cari Room \\
% \begin{table}[h]
%     \centering
%     \caption{Hasil Pengujian Cari Room}
%     \label{tb:tabel-cariroom}
%     \begin{tblr}{
%       vlines,
%       hline{1,7} = {-}{0.08em},
%       hline{2} = {-}{},
%       hline{3-6} = {4,6}{},
%     }
%     ID  & {Rincian \\Pengujian} & {Hasil yang\\di harapkan}             & Player & {Hasil yang \\didapatkan}             & Keterangan \\
%     R01 & {Tombol \\Cari Room}  & {Menampilkan \\Panel\\Pencarian Room} & 1      & {Menampilkan \\Panel\\Pencarian Room} & Berhasil   \\
%         &                       &                                       & 2      &                                       & Berhasil   \\
%         &                       &                                       & 3      &                                       & Berhasil   \\
%         &                       &                                       & 4      &                                       & Berhasil   \\
%         &                       &                                       & 5      &                                       & Berhasil   
%     \end{tblr}
%     \end{table}
%     \newpage
%     \item Tombol Buat Room \\
%     \begin{table}[h]
%         \centering
%         \caption{Hasil Pengujian Buat Room}
%         \label{tb:tabel-buatroom}
%         \begin{tblr}{
%           vlines,
%           hline{1,7} = {-}{0.08em},
%           hline{2} = {-}{},
%           hline{3-6} = {4,6}{},
%         }
%         ID  & {Rincian \\Pengujian} & {Hasil yang\\di harapkan}                & Player & {Hasil yang \\didapatkan}               & Keterangan \\
%         B01 & {Tombol\\Buat Room}   & {Menampilkan \\Panel Inputan\\Nama Room} & 1      & {Menampilkan\\Panel Inputan\\Nama Room} & Berhasil   \\
%             &                       &                                          & 2      &                                         & Berhasil   \\
%             &                       &                                          & 3      &                                         & Berhasil   \\
%             &                       &                                          & 4      &                                         & Berhasil   \\
%             &                       &                                          & 5      &                                         & Berhasil   
%         \end{tblr}
%         \end{table}
%         \newpage
%         \item Tombol Exit Room \\
%         \begin{table}[h]
%             \centering
%             \caption{Hasil Pengujian Exit Room}
%         \label{tb:tabel-exitroom}
%             \begin{tblr}{
%               vlines,
%               hline{1,7} = {-}{0.08em},
%               hline{2} = {-}{},
%               hline{3-6} = {4,6}{},
%             }
%             ID  & {Rincian \\Pengujian} & {Hasil yang\\di harapkan}  & Player & {Hasil yang \\didapatkan} & Keterangan \\
%             E01 & {Tombol\\Exit Room}   & {Menampilkan~\\Menu Utama} & 1      & {Menampilkan\\Menu utama} & Berhasil   \\
%                 &                       &                            & 2      &                           & Berhasil   \\
%                 &                       &                            & 3      &                           & Berhasil   \\
%                 &                       &                            & 4      &                           & Berhasil   \\
%                 &                       &                            & 5      &                           & Berhasil   
%             \end{tblr}
%             \end{table}
%         \item Tombol About Game \\
%         \begin{table}[h]
%             \centering
%             \caption{Hasil Pengujian About Game}
%             \label{tb:tabel-aboutgame}
%             \begin{tblr}{
%               vlines,
%               hline{1,7} = {-}{0.08em},
%               hline{2} = {-}{},
%               hline{3-6} = {4,6}{},
%             }
%             ID  & {Rincian \\Pengujian} & {Hasil yang\\di harapkan} & Player & {Hasil yang \\didapatkan} & Keterangan \\
%             A01 & {Tombol\\About\\Game} & {Menampilkan\\Informasi}  & 1      & {Menampilkan\\Informasi}  & Berhasil   \\
%                 &                       &                           & 2      &                           & Berhasil   \\
%                 &                       &                           & 3      &                           & Berhasil   \\
%                 &                       &                           & 4      &                           & Berhasil   \\
%                 &                       &                           & 5      &                           & Berhasil   
%             \end{tblr}
%             \end{table}
%         \newpage
%         \item Tombol Keluar \\
%         \begin{table}[h]
%             \centering
%             \caption{Hasil Pengujian Keluar Game}
%             \label{tb:tabel-keluargame}
%             \begin{tblr}{
%               vlines,
%               hline{1,7} = {-}{0.08em},
%               hline{2} = {-}{},
%               hline{3-6} = {4,6}{},
%             }
%             ID  & {Rincian \\Pengujian} & {Hasil yang\\di harapkan} & Player & {Hasil yang \\didapatkan} & Keterangan \\
%             K01 & {Tombol\\Keluar}      & {Mengeluarkan\\Permainan} & 1      & {Mengeluarkan\\Permainan} & Berhasil   \\
%                 &                       &                           & 2      &                           & Berhasil   \\
%                 &                       &                           & 3      &                           & Berhasil   \\
%                 &                       &                           & 4      &                           & Berhasil   \\
%                 &                       &                           & 5      &                           & Berhasil   
%             \end{tblr}
%             \end{table}
% \end{enumerate}

% Pada Tabel \ref{tb:tabel-blackbox} untuk menampilkan menghitung persentase kelayakan menu game dari 5 pengguna.
% \begin{table}[h]
%     \centering
%     \caption{Hasil Pengujian Black Box}
%     \label{tb:tabel-blackbox}
%     \begin{tabular}{|ll|l|l|l|} 
%     \toprule
%     \multicolumn{1}{|l|}{\begin{tabular}[c]{@{}l@{}}Test\\Case\\Id\end{tabular}} & Penggunaan Berhasil                      & \begin{tabular}[c]{@{}l@{}}Pengunaan\\Tidak\\Berhasil\end{tabular} & Presentase & Hasil                 \\ 
%     \hline
%     R01                                                                          & 5                                        & 0                                                                  & $5/5 \times 100\%$  & 100\%                      \\ 
%     \hline
%     B01                                                                          & 5                                        & 0                                                                  & $5/5 \times 100\%$  & 100\%                      \\ 
%     \hline
%     E01                                                                          & 5                                        & 0                                                                  & $5/5 \times 100\%$  & 100\%                      \\ 
%     \hline
%     A01                                                                          & 5                                        & 0                                                                  & $5/5 \times 100\%$  & 100\%                      \\ 
%     \hline
%     K01                                                                          & 5                                        & 0                                                                  & $5/5 \times 100\%$  & 100\%                      \\ 
%     \hline
%                                                                                  & \multicolumn{1}{l}{Rata Rata Presentase} & \multicolumn{1}{l}{}                                               &            & 100\%  \\
%     \bottomrule
%     \end{tabular}
%     \end{table}

%     Berdasarkan tabel atas dijelaskan pengujian kelayakan menu utama game menggunakan 
% metode black box dengan 5 pengguna didapatkan hasil 100\% dari pengguna 
% berhasil.
\newpage
\subsection{Uji Coba Sinkronisasi Permainan}
\noindent

Uji coba ini berlangsung dengan bermain langsung menggunakan berbagai perangkat android. Ketika pemain menggerakan karakternya, secara \textit{remote} karakter tersebut menggerakannya di klien lainnya, sehingga klien lain dapat melihat pergerakan yang dilakukan oleh pemain lawannya tersebut. Koneksi internet berpengaruh pada proses sinkronisasi, koneksi yang lambat akan mempengaruhi koneksi yang lancar, maka ketika antar klien saling menembakkan, atribut health akan 
terkalkulasi dengan prediksi oleh Photon, sehingga pemain akan 
merasakan keanehan ketika dia berhasil menembakkan projectile
ke lawan, terkadang tidak mengurangi atribut health lawan karena 
mungkin tidak sesuai dengan posisi sebenarnya.

\begin{table}[h]
    \centering
    \caption{Pengujian Sinkronisasi Karakter}
    \label{tb:tabel-karakters}
    \begin{tabular}{|l|l|}
    \hline
    \multicolumn{1}{|c|}{\textbf{Nama Pengujian}}                             & \multicolumn{1}{c|}{Sinkronisasi Pergerakan}                                                                                             \\ \hline
    \textbf{Tujuan}                                                           & \begin{tabular}[c]{@{}l@{}}Melakukan sinkronisasi antar \\ klien pada setiap perubahan posisi\\ objek karakter\end{tabular}              \\ \hline
    \textbf{\begin{tabular}[c]{@{}l@{}}Prosedur\\ Pengujian\end{tabular}}     & \begin{tabular}[c]{@{}l@{}}Salah satu klien pemilik objek\\ melakukan pergerakkan \\ menggunakan virtual analog\end{tabular}             \\ \hline
    \textbf{\begin{tabular}[c]{@{}l@{}}Hasil Yang \\ Diharapkan\end{tabular}} & \begin{tabular}[c]{@{}l@{}}Pada setiap klien yang bukan \\ pemilik objek akan melakukan\\ remote pergerakkan objek karakter\end{tabular} \\ \hline
    \textbf{Pengujian}                                                        & \multicolumn{1}{c|}{\textbf{Berhasil}}                                                                                                   \\ \hline
    \end{tabular}
    \end{table}

    \newpage
    \begin{table}[h]
        \centering
        \caption{Pengujian Sinkronisasi Darah}
        \label{tb:tabel-darah}
        \begin{tabular}{|l|l|}
        \hline
        \multicolumn{1}{|c|}{\textbf{Nama Pengujian}}                             & \multicolumn{1}{c|}{Sinkronisasi Pengurangan Darah}                                                                                              \\ \hline
        \textbf{Tujuan}                                                           & \begin{tabular}[c]{@{}l@{}}Melakukan sinkronisasi saat client\\ menembaki client lainnya healthbarnya berkurang\end{tabular}                                            \\ \hline
        \textbf{\begin{tabular}[c]{@{}l@{}}Prosedur\\ Pengujian\end{tabular}}     & \begin{tabular}[c]{@{}l@{}}Salah satu klien menembaki client\\ \\ lainnya.\end{tabular}                                                          \\ \hline
        \textbf{\begin{tabular}[c]{@{}l@{}}Hasil Yang \\ Diharapkan\end{tabular}} & \begin{tabular}[c]{@{}l@{}}Jika arah aiming sesuai dengan collider\\ client, maka client yang ditembaki akan \\ berkurang darahnya.\end{tabular} \\ \hline
        \textbf{Pengujian}                                                        & \multicolumn{1}{c|}{\textbf{Berhasil}}                                                                                                           \\ \hline
        \end{tabular}
        \end{table}
\newpage
        \begin{table}[h]
            \centering
            \caption{Pengujian Sinkronisasi \textit{kill}}
            \label{tb:tabel-kill}
            \begin{tabular}{|l|l|}
            \hline
            \multicolumn{1}{|c|}{\textbf{Nama Pengujian}}                             & \multicolumn{1}{c|}{Sinkronisasi \textit{kill}}                                                                                         \\ \hline
            \textbf{Tujuan}                                                           & \begin{tabular}[c]{@{}l@{}}Melakukan sinkronisasi saat client\\ menembaki client lainnya akan menambahkan\\ kill\end{tabular} \\ \hline
            \textbf{\begin{tabular}[c]{@{}l@{}}Prosedur\\ Pengujian\end{tabular}}     & \begin{tabular}[c]{@{}l@{}}Salah satu klien menembaki client\\ \\ lainnya sampai darahnya habis.\end{tabular}                  \\ \hline
            \textbf{\begin{tabular}[c]{@{}l@{}}Hasil Yang \\ Diharapkan\end{tabular}} & \begin{tabular}[c]{@{}l@{}}Akan menambahkan score pada leaderboard \\ interface\end{tabular}                                   \\ \hline
            \textbf{Pengujiann}                                                        & \multicolumn{1}{c|}{\textbf{Berhasil}}                                                                                         \\ \hline
            \end{tabular}
            \end{table}

            \begin{table}[h]
                \centering
            \caption{Pengujian Sinkronisasi \textit{Death}}
            \label{tb:tabel-death}
                \begin{tabular}{|l|l|}
                \hline
                \multicolumn{1}{|c|}{\textbf{Nama Pengujian}}                             & \multicolumn{1}{c|}{Sinkronisasi Death}                                                                                  \\ \hline
                \textbf{Tujuan}                                                           & \begin{tabular}[c]{@{}l@{}}Melakukan sinkronisasi saat client \textit{death}\\ client lainnya.\end{tabular}                   \\ \hline
                \textbf{\begin{tabular}[c]{@{}l@{}}Prosedur\\ Pengujian\end{tabular}}     & \begin{tabular}[c]{@{}l@{}}Salah satu klien berdiam diri untuk ditembaki\\ oleh client lainnya\end{tabular}              \\ \hline
                \textbf{\begin{tabular}[c]{@{}l@{}}Hasil Yang \\ Diharapkan\end{tabular}} & \begin{tabular}[c]{@{}l@{}}client yang berdiam diri akan terdestroy objectnya\\ dan ui death akan terupdate\end{tabular} \\ \hline
                \textbf{Pengujian}                                                        & \multicolumn{1}{c|}{\textbf{Berhasil}}                                                                                   \\ \hline
                \end{tabular}
                \end{table}
    \newpage
    \begin{table}[h]
        \centering
        \caption{Pengujian Sinkronisasi \textit{Round Over}}
        \label{tb:tabel-roundover}
        \begin{tabular}{|l|l|}
        \hline
        \multicolumn{1}{|c|}{\textbf{Nama Pengujian}}                             & \multicolumn{1}{c|}{Sinkronisasi Round Over}                                                                                            \\ \hline
        \textbf{Tujuan}                                                           & \begin{tabular}[c]{@{}l@{}}Melakukan sinkronisasi saat permainan telah\\ \\ selesai\end{tabular}                                        \\ \hline
        \textbf{\begin{tabular}[c]{@{}l@{}}Prosedur\\ Pengujian\end{tabular}}     & \begin{tabular}[c]{@{}l@{}}pemain bermain sesuai dengan waktu yang\\ telah ditentukan\end{tabular}                                      \\ \hline
        \textbf{\begin{tabular}[c]{@{}l@{}}Hasil Yang \\ Diharapkan\end{tabular}} & \begin{tabular}[c]{@{}l@{}}akan menampilkan panel round over dan\\ menunggu 10 detik akan dialihkan ke\\ scene menu utama.\end{tabular} \\ \hline
        \textbf{Pengujian}                                                        & \multicolumn{1}{c|}{\textbf{Berhasil}}                                                                                                  \\ \hline
        \end{tabular}
        \end{table}
    Pada uji coba ini aplikasi permainan berhasil melakukan sinkronisasi pergerakan, \textit{health}, \textit{kill}, \textit{death} dan waktu permainan. Dapat dilihat pada gambar \ref{fig:pergerakan} sebagai sinkronisasi client, gambar \ref{fig:leaderkill} sebagai sinkronisasi leaderboard \textit{kill/death}, gambar \ref{fig:killed} sebagai sinkronisasi saat di tembaki dan gambar \ref{fig:roundover2} saat waktu permainan yang ditentukan telah habis.
    \begin{figure}[h]
        \centering
        \includegraphics[width=8cm]{pergerakan.jpg}
        \caption{Tampilan Sinkronisasi Pergerakan}
        \label{fig:pergerakan}
    \end{figure}
    \newpage
    \begin{figure}[h]
        \centering
        \includegraphics[width=10cm]{leaderboard.jpg}
        \caption{Tampilan Sinkronisasi \textit{Kill} Dan \textit{Death}}
        \label{fig:leaderkill}
    \end{figure}
    \begin{figure}[h]
        \centering
        \includegraphics[width=10cm]{killed.jpg}
        \caption{Tampilan Sinkronisasi \textit{killed}}
        \label{fig:killed}
    \end{figure}
    \newpage
    \begin{figure}[h]
        \centering
        \includegraphics[width=10cm]{roundover.png}
        \caption{Tampilan Sinkronisasi \textit{Round Over}}
        \label{fig:roundover2}
    \end{figure}

    \subsection{Pengujian matchmaking}
    \noindent

    Pada uji coba ini menggunakan beberapa klien yang bersamaan memilih pencarian room / Pembuatan room. Aplikasi permainan harus terhubung ke internet terlebih dahulu agar dapat menggunakan fiture ini.
\begin{enumerate}
    \item Pengujian Mencari room \\
    Pada Pengujian ini player pertama mencari room yang dibuat oleh player kedua, dan pada pengujian pencarian room ini berhasil seperti gambar \ref{fig:pencarianroom}. 
    \begin{figure}[h]
        \centering
        \includegraphics[width=10cm]{pencarianroom.png}
        \caption{Tampilan Pencarian Room}
        \label{fig:pencarianroom}
    \end{figure}
    \item Pengujian Masuk Room \\
    Pada Pengujian memasuki room player berhasil menjumpai sesama player didalam room tersebut seperti gambar \ref{fig:didalamroom}.
    \begin{figure}[h]
        \centering
        \includegraphics[width=10cm]{5-player.png}
        \caption{Tampilan Didalam Room}
        \label{fig:didalamroom}
    \end{figure}
\end{enumerate}
\subsection{Pengujian QOS \textit{(Quality Of Service)}}
\noindent

Pengukuran QOS dilakukan untuk mengetahui performa jaringan dari photon unity networking.
\begin{enumerate}
    \item \textit{Packet Loss} \\
    Packet loss merupakan banyaknya paket yang gagal 
mencapai tempat tujuan paket tersebut dikirim. Ketika packet 
loss besar maka dapat diketahui bahwa jaringan sedang sibuk 
atau terjadi overload. Berdasarkan teori dan hasil pengamatan 
disaat melakukan pengukuran ,maka hasil yang didapatkan 
bisa dilihat pada \ref{tb:tabel-packetloss}.
\newpage
\begin{table}[h]
    \centering
    \caption{Hasil Pengukuran Packet loss}
    \label{tb:tabel-packetloss}
    \begin{tabular}{|c|c|c|c|c|} 
    \hline
    No & \begin{tabular}[c]{@{}c@{}}Hari\\Tanggal\end{tabular}        & Pengujian & Packet Loss & Tiphon        \\ 
    \hline
    1  & \begin{tabular}[c]{@{}c@{}}Rabu\\26 July\\2023\end{tabular}  & 1         & 0,1\%       & Sangat Bagus  \\ 
    \hline
    2  & \begin{tabular}[c]{@{}c@{}}Rabu\\26 July\\2023\end{tabular}  & 2         & 0\%         & Sangat Bagus  \\ 
    \hline
    3  & \begin{tabular}[c]{@{}c@{}}Rabu\\26 July \\2023\end{tabular} & 3         & 0,6\%       & Sangat Bagus  \\ 
    \hline
    4  & \begin{tabular}[c]{@{}c@{}}Rabu\\26~July\\2023\end{tabular}  & 4         & 0,2\%       & Sangat Bagus  \\ 
    \hline
    5  & \begin{tabular}[c]{@{}c@{}}Rabu\\26~July\\2023\end{tabular}  & 5         & 0\%         & Sangat Bagus  \\
    \hline
    \end{tabular}
    \end{table}

    Dari hasil pengukuran packet loss menurut standar TIPHON jika rata-rata packet loss 
0 maka masuk kedalam kategori “Sangat Bagus”.
\newpage
    \item \textit{Delay} \\
    Delay merupakan lamanya waktu yang dibutuhkan oleh 
    data atau informasi untuk sampai ke tempat tujuan data atau 
    informasi tersebut dikirim. Delay pada suatu jaringan akan 
    menentukan langkah apa yang akan kita ambil ketika kita 
    memanajemen suatu jaringan. Berdasarkan teori dan hasil 
    pengamatan disaat melakukan pengukuran ,maka hasil yang 
    didapatkan bisa dilihat pada \ref{tb:tabel-delay}
    
    \begin{table}[h]
        \centering
        \caption{Hasil Pengukuran Delay}
        \label{tb:tabel-delay}
        \begin{tabular}{|c|c|c|c|c|} 
        \hline
        No & \begin{tabular}[c]{@{}c@{}}Hari\\Tanggal\end{tabular}        & Pengujian & \begin{tabular}[c]{@{}c@{}}Delay\\Average\end{tabular} & Tiphon        \\ 
        \hline
        1  & \begin{tabular}[c]{@{}c@{}}Rabu\\26 July\\2023\end{tabular}  & 1         & 125                                                    & Sangat Bagus  \\ 
        \hline
        2  & \begin{tabular}[c]{@{}c@{}}Rabu\\26 July\\2023\end{tabular}  & 2         & 110                                                    & Sangat Bagus  \\ 
        \hline
        3  & \begin{tabular}[c]{@{}c@{}}Rabu\\26 July \\2023\end{tabular} & 3         & 125                                                    & Sangat Bagus  \\ 
        \hline
        4  & \begin{tabular}[c]{@{}c@{}}Rabu\\26~July\\2023\end{tabular}  & 4         & 90                                                     & Sangat Bagus  \\ 
        \hline
        5  & \begin{tabular}[c]{@{}c@{}}Rabu\\26~July\\2023\end{tabular}  & 5         & 130                                                    & Sangat Bagus  \\
        \hline
        \end{tabular}
        \end{table}
    
        Dari hasil pengukuran delay untuk masing-masing 
    pengguna adalah tertinggi terdapat di 1,3 dan 5 pengguna menurut standar TIPHON jika 
    rata-rata delay dibawah 150 ms maka masuk kedalam kategori 
    “Sangat Bagus”.
\end{enumerate}
