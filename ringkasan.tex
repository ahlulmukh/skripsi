\chapter*{ABSTRAK}
\addcontentsline{toc}{chapter}{RINGKASAN}
\noindent


Perkembangan teknologi game pada software dan hardware, khususnya di perangkat PC, terus berkembang pesat setiap tahunnya. Kini, aplikasi game dalam skala 3D dapat dijalankan dengan lancar, dan integrasi antara software dan hardware semakin sempurna. Kinerja optimal menciptakan pengalaman bermain yang bervariasi bagi pengguna. Dalam peningkatan pengalaman bermain game, bermain secara bersamaan dengan pemain lain menjadi salah satu cara untuk meningkatkan keseruan. Skripsi ini membangun game bergenre FPS dengan gameplay interaktif, memungkinkan karakter yang dapat dikontrol, dan mengimplementasikan synchronous multiplayer. Dalam game tersebut, pemain dapat berinteraksi secara real-time dengan pemain lainnya, menciptakan suasana yang menyenangkan. Metode yang digunakan adalah kombinasi game engine Unity dengan framework khususnya, yaitu Photon Unity Networking (PUN), yang berfungsi untuk menyatukan pemain secara real-time. Semua ini menghadirkan pengalaman bermain game yang lebih mengasyikkan dan seru bagi para pengguna.
\newline \break
\noindent Kata Kunci: FPS Multiplayer, Online Multiplayer, Photon Unity Networking.